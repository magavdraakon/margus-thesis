\chapter{Analysis}
\label{analysis}
Analysis...


\section{Problem Analysis}
\label{Problem Analysis}
Problem Analysis...

\section{Related Work}
\label{Related Work}
Related Work...




\subsubsection{Random tags}
\begin{itemize}
	\item normal traffic generator
	\item malicious traffic generator
	\item availability monitor (for grading)
\end{itemize}

\section{Specialness of the e-learning}
\begin{itemize}
	\item Millal e-õpe töötab ja millal ei? Õppur peab rohkem pingutama [Chao: 11]
	\item Erinevad õpikeskkonnad ja nende sobivus antud õppeks
		\begin{itemize}
			\item Moodle
			\item Blackboard WebCT
			\item CISCO Network Academy
			\item Maurus
			\item IVA
			\item Sakai
			\item Wikiversity
			\item TUT Kaur course lab
		\end{itemize}
	\item virtual distance laboratory
	\item security aspects of distance laboratory
\end{itemize}

\section{Developing an e-course}
\begin{itemize}
	\item Analysis
	\begin{itemize}
		\item Analysing of the requirements and establishing goals for e-course
		\item Analysing of restrictions and scope
		\item Analysis of target group
		\item Analysis of content of the course
	\end{itemize}
	\item Planning of learning process
	\begin{itemize}
		\item Pedagogical view of e-course
		\item Planning grading techniques
		\item Choosing technological tools
	\end{itemize}
	\item Implementation of the e-course
	\begin{itemize}
		\item Developing learning material
		\item Text based learning material
		\item Audio/Visual learning material
		\item Interactive learning material
		\item Online tests and laboratory scenarios 
		\item õpijuhis 
		\item technical implementaion
		\item testing course
	\end{itemize}
	\item Kursuse läbiviimine
	\begin{itemize}
		\item Organizational role
		\item Social role
		\item Pedagogical role
	\end{itemize}
\end{itemize}

\section{Cyber security aspects of the e-learning}
