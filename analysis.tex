\chapter{Analysis}
\label{analysis}
In this chapter ... 
\url{http://www.qatar.cmu.edu/iliano/courses/12F-CMU-CS349/index.php?page=info} - uurida


\section{Problem Analysis}
\label{Problem Analysis}


Study program focuses on practical learning by doing approach to ITC subjects. The students gain skills and knowledge as they progress through curricula lectures and practical classes. The studies include lectures, practical classes and independent work as homework. By and large, one subject is divided as follows: 25\% lectures, 25\% practical classes and 50\% homework which is mostly working with materials (books, web articles). In the worst case, practical work constitutes only 25\% and takes place at \gls{EITC} computer classes. Students are not interested in learning mere theory. The formulas are not seen necessary nor linked to their study area or future job. Theory that is not used will be forgotten quickly. In a few years students won't even remember if a specific topic was covered or not. Applied education should introduce practical approach and learning by doing. In ITC field practical classes and practical homework is the key to achieving acceptable results. Students appreciate learning IT system administration by building complex systems using virtual distance laboratory environment and gaining practical experience during this process following learning-by-doing study model.
The problem with practical classes in the cyber field that the practical approach requires hardware, software and preconfigured laboratory environment. However, the  distance laboratory environment can be developed in a manner  that practical classes can be accessed from any Internet connected computer using web browser. Moreover, using virtual and game-like environments is a contemporary approach for teaching IT System administration and programming focusing to the cyber security requirements. The outcome of the developed environment and contemporary teaching methods are students with increased motivation and skills in the cyber field. The tasks given to the students for solving have to be associated with the practical applicability of the theoretical knowledge.

Developed systems allows to  increase of the proportion of practical work by converting independent and theoretical homework to  the practical classes. The quality of studies will improve due to increased amount of practical hands-on classes. The developed system uses only open source components and released using open source compatible MIT licence. The development and hardware are funded by European Social Fund (\gls{ESF}). We expect to have results of using virtual laboratory system to be available in 2013.



Problem Analysis...
\begin{itemize}
	\item Küberjulgeoleku strateegia 2008-2013
	\item RIA sisend EIK-le
	\item EIK õppekavad pole turva poolelt muutunud peale nende tegemist
\end{itemize}

\section{Related Work}
\label{Related Work}
Related Work...
Panna siia lühikokkuvõte loetud teadusartikklitest IEEE*




\subsubsection{Random tags}
\begin{itemize}
	\item normal traffic generator
	\item malicious traffic generator
	\item availability monitor (for grading)
\end{itemize}

\section{Specialness of the e-learning}
\begin{itemize}
	\item Millal e-õpe töötab ja millal ei? Õppur peab rohkem pingutama [Chao: 11]
	\item Erinevad õpikeskkonnad ja nende sobivus antud õppeks
		\begin{itemize}
			\item Moodle
			\item Blackboard WebCT
			\item CISCO Network Academy
			\item Maurus
			\item IVA
			\item Sakai
			\item Wikiversity
			\item TUT Kaur course lab
		\end{itemize}
	\item virtual distance laboratory
	\item security aspects of distance laboratory
\end{itemize}

\section{Choosing Methodology for Developing an e-course}



The methodology used to develop this e-course should encourage student activity in learning process. Moreover student should have possibility to choice learning speed, -place and -time. Today's students have different learning style and background and methodology should reckon with individual differences...

Developed e-course should support people with disabilities. In \gls{EITC} several people have hearing disabilities and all important material should presented also without audio. For example in screen casts videos all important information should be written also in screen or added as transcript.

Today’s learning environment should support student communities where students can act as mentors and also feel part on the study program. Course integration with student driven initiative like forums, blogs, wiki pages and other collaboration learning methods should be possible and not restricted.

\subsection{E-course design methodologies}

Several methodologies are suitable to develop e-courses. \gls{ADDIE Model}

Methodology 
ADDIE is short for Analyze, Design, Develop, Implement, and Evaluate \url{http://ed.isu.edu/addie/index.html}
\url{http://www.regent.edu/admin/ctl/coursedesign/home.cfm}
Weaknesses of the ADDIE Model \url{http://www.instructionaldesign.org/models/addie_weaknesses.html}
\url{http://www.learningsolutionsmag.com/articles/1012/}

\url{http://rjh.goingeast.ca/2011/07/05/design-research-for-online-learning-course-design-edumooc/} - uurida

\section{Alalysis of the e-course}
\subsection{Analysis of target group}
The target group are second and third years students who already mastered basics of operating systems.
The System Administrator's Job

\subsection{Analysing of the requirements and establishing goals for e-course}
\subsection{Analysing of restrictions and scope}
\subsection{Analysis of content of the course}
\section{Planning of learning process}
\subsection{Pedagogical view of e-course}
\subsection{Planning grading techniques}
\subsection{Choosing technological tools}

\section{Implementation of the e-course}

\subsection{Developing learning material}
\subsection{Text based learning material}
\subsection{Audio/Visual learning material}
\subsection{Interactive learning material}
\subsection{Online tests and laboratory scenarios}
\subsection{õpijuhis }
\subsection{technical implementaion}
\subsection{testing course}
\section{Kursuse läbiviimine}
\subsection{Organizational role}
\subsection{Social role}
\subsection{Pedagogical role}

\section{Cyber security aspects of the e-learning}
