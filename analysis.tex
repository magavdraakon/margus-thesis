\chapter{Analysis}
\label{analysis}
Analysis...
\url{http://www.qatar.cmu.edu/iliano/courses/12F-CMU-CS349/index.php?page=info} - uurida

\section{Problem Analysis}
\label{Problem Analysis}
Problem Analysis...
\begin{itemize}
	\item Küberjulgeoleku strateegia 2008-2013
	\item RIA sisend EIK-le
	\item EIK õppekavad pole turva poolelt muutunud peale nende tegemist
\end{itemize}

\section{Related Work}
\label{Related Work}
Related Work...
Panna siia lühikokkuvõte loetud teadusartikklitest IEEE*




\subsubsection{Random tags}
\begin{itemize}
	\item normal traffic generator
	\item malicious traffic generator
	\item availability monitor (for grading)
\end{itemize}

\section{Specialness of the e-learning}
\begin{itemize}
	\item Millal e-õpe töötab ja millal ei? Õppur peab rohkem pingutama [Chao: 11]
	\item Erinevad õpikeskkonnad ja nende sobivus antud õppeks
		\begin{itemize}
			\item Moodle
			\item Blackboard WebCT
			\item CISCO Network Academy
			\item Maurus
			\item IVA
			\item Sakai
			\item Wikiversity
			\item TUT Kaur course lab
		\end{itemize}
	\item virtual distance laboratory
	\item security aspects of distance laboratory
\end{itemize}

\section{Choosing Methodology for Developing an e-course}



The methodology used to develop this e-course should encourage student activity in learning process. Moreover student should have possibility to choice learning speed, -place and -time. Today's students have different learning style and background and methodology should reckon with individual differences...

Developed e-course should support people with disabilities. In \gls{EITC} several people have hearing disabilities and all important material should presented also without audio. For example in screen casts videos all important information should be written also in screen or added as transcript.

Today’s learning environment should support student communities where students can act as mentors and also feel part on the study program. Course integration with student driven initiative like forums, blogs, wiki pages and other collaboration learning methods should be possible and not restricted.

E-course design methodologies

Methodology 
ADDIE is short for Analyze, Design, Develop, Implement, and Evaluate \url{http://ed.isu.edu/addie/index.html}
\url{http://www.regent.edu/admin/ctl/coursedesign/home.cfm}
Weaknesses of the ADDIE Model \url{http://www.instructionaldesign.org/models/addie_weaknesses.html}
\url{http://www.learningsolutionsmag.com/articles/1012/}

\url{http://rjh.goingeast.ca/2011/07/05/design-research-for-online-learning-course-design-edumooc/} - uurida

\begin{itemize}
	\item Analysis
	\begin{itemize}
		\item Analysing of the requirements and establishing goals for e-course
		\item Analysing of restrictions and scope
		\item Analysis of target group
		\item Analysis of content of the course
	\end{itemize}
	\item Planning of learning process
	\begin{itemize}
		\item Pedagogical view of e-course
		\item Planning grading techniques
		\item Choosing technological tools
	\end{itemize}
	\item Implementation of the e-course
	\begin{itemize}
		\item Developing learning material
		\item Text based learning material
		\item Audio/Visual learning material
		\item Interactive learning material
		\item Online tests and laboratory scenarios 
		\item õpijuhis 
		\item technical implementaion
		\item testing course
	\end{itemize}
	\item Kursuse läbiviimine
	\begin{itemize}
		\item Organizational role
		\item Social role
		\item Pedagogical role
	\end{itemize}
\end{itemize}

\section{Cyber security aspects of the e-learning}
