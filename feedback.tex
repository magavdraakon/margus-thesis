\chapter{Evaluation of the E-learning Course}
\label{Evaluation of the E-learning Course}
\lettrine[lraise=0.1, nindent=0em, slope=-.5em]{\color{Violet}T}{he}  evaluation phase of the ADDIE Model...
\begin{enumerate}
\item Formative Evaluation
\item Summative Evaluation
\end{enumerate}

Each part of the processes step needs evaluation - this is formative evaluation.

1. one-to-one - üks sihtgrupist, küsi clarity, impact - kas oli abiks, feasability - kui praktiline (enne tuleb assessment questions kirja panna)
2. small group  - different subgroups in group asses clearity,impact, feasibility. Näiteküsimused: Kas juhend oli huvitav= Kas juhend oli arusaadav? Kas materjalid olid seotud väljunditega, kas said piisavalt tagasisidet
3. field trial -  real-time rehersal (assess clarity,impact, feasibility.


\section{Feedback from Students and Lecturers}

During evaluation the learning materials tested by four different groups:

\begin{itemize}
\item Foreign system administrators (mostly GNU/Linux and network administrators)
\item Estonian IT system administrators (with different backgrounds)
\item International students during Intensive Programme
\item Estonian Students and Distance learners
\end{itemize}


All groups assessed the course very valuable and interesting. Most of the students stated, that course was too hard (because lack of GNU/Linux knowledge). Most of the students valued good balance between practical work and theoretical reading/(video)lectures. For many people the learning materials where too difficult to follow during time given. Therefore author proposed to rise \gls{ECTS} for course from 5 points  to 6 \gls{ECTS} points.



During pilot courses in fall 2012 and spring 2013 standard course feedback collected from students. 

Feedback from students (feedback from Study Information System)
Grade for course  (4.9 -- distance learners, 4.6 -- students, max is 5) 
Grade for lecturer (4.9 -- distance learners, 4.8 -- students)
Feedback from continuous education students
Grade for course (2.9 max is 3)
Grade for lecturer (2.9 max is 3)



Feedback from two lecturers, one \gls{EITC} and other from Vaasa University of Applied Sciences (web application security labs)
Too intensive to so limited time
Too much work (preparing for lab needs work before every course)


Summative evaluation - prove the worth of evaluation
\begin{enumerate}
\item reaction to question (instructions where clear and easy to understand to me?) (strongly disagree, disagree, neutral, agree, strongly agree) few open-ended questions - about owerall strength and weaknesses of the course NB anonymous feedback
\item learning - typicaly Post-test and achievement test. skills - performance test, attitudes -questionaries
\item behaviour - kas kasutatakse seda teadmiseid
\item results profits, productivity, morale
\end{enumerate}

Grade for course  (4.858 – distance learners, 4.6 – students)
Grade for lecturer (4.88 – distance learners, 4.8 - students) \footnote{\url{https://wiki.itcollege.ee/images/2/28/It-infra-tagasiside-2012-2.pdf}}\footnote{\url{https://wiki.itcollege.ee/images/4/4d/It-infra-tagasiside-2012-1.pdf}}



