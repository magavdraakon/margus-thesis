\chapter{Evaluation of the E-learning Course}
\label{Evaluation of the E-learning Course}
\lettrine[lraise=0.1, nindent=0em, slope=-.5em]{\color{Violet}T}{he} \gls{ADDIE} model suggests two evaluation methods as: formative evaluation and summative evaluation~\citep{OppeArenduskeskus2010}. The formative evaluation is done during the process as evaluation of the each phase of the process as seen in tables: ~\ref{tab:evoluation_analysis},~\ref{table:desing_develop_evaluation},~\ref{table:implementation_evaluation}. 
The evaluation information for development and design phase is presented in one table~\ref{table:desing_develop_evaluation}.

The summative evaluation contains the testing of students (do they acived the learning outcomes) and feedback from the students and lecturers.
\section{Feedback from Students and Lecturers}
The feedback collected from students during Fall semester 2012 and Spring semester 2013 and from system system administrators
During the pilot courses with different target groups 
During evaluation the learning materials tested by four different groups:

\begin{itemize}
\item Foreign system administrators (mostly GNU/Linux and network administrators)
\item Estonian IT system administrators (with different backgrounds)
\item International students during Intensive Programme
\item Estonian Students and Distance learners
\end{itemize}

All groups assessed the course very valuable and interesting. Most of the students stated, that course was too hard (because lack of GNU/Linux knowledge). Most of the students valued good balance between practical work and theoretical reading/(video)lectures. For many people the learning materials where too difficult to follow during time given. Therefore author proposed to rise \gls{ECTS} for course from 5 points  to 6 \gls{ECTS} points.

During pilot courses in  2012 standard course feedback collected from students. From 34 studentst answered 17 student (50\%) and grades is in scale 1---5. 

om students (feedback from Study Information System)

Maximum grade is 5 and feedback give 17 students.
Grade for course is 4.858 from distance learners and  4.6 from students on five point scale.
Grade for lecturer 4.88 from distance learners and 4.8 from students, on five point scale
.\footnote{\url{https://wiki.itcollege.ee/images/2/28/It-infra-tagasiside-2012-2.pdf}}\footnote{\url{https://wiki.itcollege.ee/images/4/4d/It-infra-tagasiside-2012-1.pdf}}

Average grades in \gls{EITC} are 4.3 for lecturers and 4.2 for subjects according to unpublished internal SELF-EVALUATION REPORT 2013.

Feedback from 50 continuous education students.
Grade for course (2.9 max is 3)
Grade for lecturer (2.9 max is 3)

Feedback from two lecturers, one from \gls{EITC} and other from Vaasa University of Applied Sciences (web application security labs)
Too intensive to so limited time.
Too much work (preparing for lab needs work before every course)


%http://www.creativeagni.com/instructional-design-articles/instructional-design-model-ADDIE-for-course-training-development.html

Although, the students found this course too hard was overall feedback positive and over \gls{EITC} average.