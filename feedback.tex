\chapter{Evaluation of the E-learning Course}
\label{Evaluation of the E-learning Course}
\lettrine[lraise=0.1, nindent=0em, slope=-.5em]{\color{Violet}T}{he} \gls{ADDIE} model suggests two evaluation methods: formative evaluation and summative evaluation~\citep{OppeArenduskeskus2010}. The formative evaluation is performed during the process as evaluation of each phase of the process as presented in tables: ~\ref{tab:evoluation_analysis},~\ref{table:desing_develop_evaluation},~\ref{table:implementation_evaluation}. 
The evaluation information concerning development and design phase is presented in table~\ref{table:desing_develop_evaluation}.

The summative evaluation is based on testing students (whether they have achieved  the learning outcomes) and feedback from the students and lecturers.

\section{Feedback from Students and Lecturers}
The feedback was collected at the end of 14 pilot courses for four different target groups in 2012 and 2013:

\begin{itemize}
\item Foreign system administrators (mostly GNU/Linux and network administrators);
\item Estonian IT system administrators (with different backgrounds);
\item International students during Intensive Programme;
\item \gls{EITC} Students and Distance learners.
\end{itemize}

All groups assessed the course as valuable and interesting. Most students stated, that course was too hard (because of  lack of GNU/Linux knowledge). Most students valued good balance between practical work and theoretical reading/(video)lectures. For many people the learning materials were too difficult to follow during the time given. Therefore author proposed  rising the credit points granted for participation from 5 points  to 6 \gls{ECTS} points.

During pilot courses in 2012 standard course feedback was collected from \gls{EITC} students in Study Information System. Out of 34 students 17 (50\%) answered  and graded the course in scale from 1 to 5.


Maximum grade given was 5. Average grade for course was 4.9 from distance learners and 4.6 from students on five point scale. The lecturer (the author of the thesis) was graded with 4.9 by distance learners and with 4.8 by students on 5-point scale~\footnote{\url{https://wiki.itcollege.ee/images/2/28/It-infra-tagasiside-2012-2.pdf}}\footnote{\url{https://wiki.itcollege.ee/images/4/4d/It-infra-tagasiside-2012-1.pdf}}.

As for comparison, average grades in \gls{EITC} are 4.3 for lecturers and 4.2 for courses according to unpublished internal SELF-EVALUATION REPORT 2013.

Course feedback was collected from 50 Estonian system administrators at the end of the pilot courses. The course was graded with 2.9 and the lecturer with 2.9 points, both in 3-point scale.


%http://www.creativeagni.com/instructional-design-articles/instructional-design-model-ADDIE-for-course-training-development.html

As for the feedback of foreign students, please see Appendix~\ref{appendix:Feedback from international students}.
Although, the students found this course too hard was overall feedback positive and over \gls{EITC} average.