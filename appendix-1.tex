

\chapter{Protecting Web Application Against (D)DOS Attacks}
\label{Protecting Web Application Against (D)DOS Attacks}

 
\begin{quote}
"If you tell me, I will listen. If you show me, I
will see. If you let me experience, I will learn." --- Lao Tzu (6th Century BC)
\end{quote}

\section{Introduction}
\section{Pre-Requirements}
\url{http://net.tutsplus.com/tutorials/tools-and-tips/http-the-protocol-every-web-developer-must-know-part-1/}
\url{http://net.tutsplus.com/tutorials/tools-and-tips/http-the-protocol-every-web-developer-must-know-part-2/}
\section{Scope}
\section{Learning Outcomes} 
\section{Setting up the Virtual Environment} 

In this lab we use two Ubuntu Linux virtual machines.
Ubuntu server 512MB RAM, NIC1 - NAT, NIC2 - HostOnly with address 192.168.56.200
Ubuntu client 1GB RAM NIC1 - NAT, NIC2 - HostOnly wid dynamic address. (probably 192.168.56.101)
Download virtual machines to local disk
\url{http://elab.itcollege.ee:8000/infra_klient_small.ova}
\url{http://elab.itcollege.ee:8000/infra_server.ova}

Import virtual machines (If your host computer has only 4GB RAM, then reduce client machine memory to 1GB)

Start both machines. 

{\small{If you got an error about host only network then open Main Menu, choose File Preferences and choose Network and add Host Only Network.}}

Username and password for both machines are student, student.

Username: student
Password: student
Student user are in sudo group and can start administrator shell with sudo command.

Log on to client and add two addresses on /etc/hosts
\begin{minted}[frame=lines,framesep=2mm]{bash}
echo "192.168.56.200 wp.planet.zz">>/etc/hosts
echo "192.168.56.200 dvwa.planet.zz">>/etc/hosts
\end{minted}

\section{Installation of the WordPress}
All following commands must executed as root user. To get root permissions in Ubuntu Server used in this lab type:


\begin{minted}[frame=lines,framesep=2mm]{bash}
sudo -i
\end{minted}



This lab demands installing software for that update local package cache first


\begin{minted}[frame=lines,framesep=2mm]{bash}
apt-get update\end{minted}

If you have time then do system upgrade
\begin{minted}[frame=lines,framesep=2mm]{bash}
apt-get dist-upgrade\end{minted}

Install apache webserver and mysql database and 
\begin{minted}[frame=lines,framesep=2mm]{bash}
apt-get install apache2 mysql-server ssh php5 php5-mysql 
apt-get install apache2-utils libapache2-mod-php5
\end{minted}

Download latest version of WordPress
\begin{minted}[frame=lines,framesep=2mm]{bash}
wget http://wordpress.org/latest.tar.gz
\end{minted}

Unpack tar.gz archive to  /var/www directory using tar utility.

\begin{minted}[frame=lines,framesep=2mm]{bash}
sudo tar zxvf latest.tar.gz --directory=/var/www/
\end{minted}

Creade new mysql database called wp and database user student. Grant all privileges on database wp to user student.

\begin{minted}[frame=lines, framesep=2mm]{bash}
mysql -u root -p
create database wp;
create user student;
GRANT ALL PRIVILEGES ON wp.* TO 'student'@'localhost' IDENTIFIED BY 'student';
quit
\end{minted}

Create new virtual host for wordpress 
\marginpar{\rule[-.9cm]{1pt}{1pt}TODO}
\begin{minted}[frame=lines,framesep=2mm]{bash}
cp /etc/apache2/sites-available/default /etc/apache2/sites-available/wp
\end{minted}

Change owner and group for wordpress files to ensure that web server can read and write files.
\begin{minted}[frame=lines,framesep=2mm]{bash}
chown www-data:www-data /var/www/wordpress -R
\end{minted}

Change root directory (DocumentRoot) for new virtualhost and add server name field (ServerName) to virtualhosts configuration file   /etc/apache2/sites-available/wp


\begin{minted}[frame=lines, framesep=2mm]{bash}
ServerName	wp.planet.zz
#DocumentRoot /var/www
DocumentRoot /var/www/wordpress
\end{minted}


To enable new virtualhost for WordPress use a2ensite utility
\begin{minted}[frame=lines,framesep=2mm]{bash}
a2ensite wp
\end{minted}

Change wordpress configuration file
/var/www/wordpress/wp-config-sample.php

Set correct values for defines DB\_NAME, DB\_USER, DB\_PASSWORD as:

define('DB\_NAME', 'wp');

/** MySQL database username */
define('DB\_USER', 'student');

/** MySQL database password */
define('DB\_PASSWORD', 'student');


Copy sample file to real config file:
\begin{minted}[frame=lines,framesep=2mm]{bash}
cp  -a /var/www/wordpress/wp-config-sample.php /var/www/wordpress/wp-config.php
\end{minted}

Reload apache configuration files:
\begin{minted}[frame=lines,framesep=2mm]{bash}
service  apache2 reload
\end{minted}

Go to address http://wp.planet.zz/ using web browser.

Enter values for  Site Title, username, password and an e-mail

Choose Install

\subsection{Testing Your WordPress Installation against simpler DOS attacks}


How many requests default installation will serve? (parallel connections, requests/second)
Install apache2 utils on CLIENT computer, not in the server computer.

\begin{minted}[frame=lines, framesep=2mm]{bash}
sudo apt-get update
sudo apt-get install apache2-utils
\end{minted}

For Fedora/CentOS/RH/Oracle Linux install httpd-utils package.

Execute Apache Benchmark program ab
\begin{minted}[frame=lines, framesep=2mm]{bash}
ab -c<NO_CONN> -t<TIME> http://wp.planet.zz/
\end{minted}
flag c - parallel connections
flag t - time for test

\begin{minted}[frame=lines,framesep=2mm]{bash}
ab -c600 -t20 http://wp.planet.zz/
\end{minted}

In last example the ab utility makes 600 parallel connections and test takes 20 seconds.
Test results
Store test results and the command line used for tests.
Write down request per second. No of failed requests and No of completed requests.

\subsection{Hardening WordPress Installation}

What is the OOM?

Disable swap (edit /etc/fstab file or use swapoff command)


\begin{minted}[frame=lines,framesep=2mm]{bash}
swapoff -a
\end{minted}

Disable OOM killer for MySQL database. In newer kernels write -1000 to oom\_score\_adj file.

\begin{minted}[frame=lines,framesep=2mm]{bash}
echo "-1000" > /proc/$(pidof mysqld)/oom_score_adj
\end{minted}
%$
For backward compatibility with old kernels (2.6.XX series) you can use oom\_adj file
\begin{minted}[frame=lines,framesep=2mm]{bash}
echo "-17" > /proc/$(pidof mysqld)/oom_adj
\end{minted}
%$
Documentation about proc filesystem and OOM:
\url{http://www.kernel.org/doc/Documentation/filesystems/proc.txt}

Not mandatory task: Modify mysql startup script to tune OOM score. 

WordPress Supercache
Install WordPress Supercache plugin.
Change Permalinks settings
Test cache with AB

Install Varnish HTTP cache
Change apache default port to 8080
In file /etc/apache2/ports.conf
Change 80 > 8080
Like:
NameVirtualHost *:8080
Listen 8080

Or just download new file using wget 

cd /etc/apache2
mv ports.conf /root/ports.conf.old
wget http://elab.itcollege.ee:8000/Configs/apache2/ports.conf

Change all virtual hosts to use new 8080 port using text editor or sed command.

\begin{minted}[frame=lines,framesep=2mm]{bash}
sed 's/:80>/:8080>/' -i /etc/apache2/sites-enabled/wp
\end{minted}


Install varnish and change varnish default port from 6081 to 80
apt-get install varnish
change /etc/default/varnish configuration file
Change line
\begin{minted}[frame=lines,framesep=2mm]{bash}
DAEMON_OPTS="-a *:6081 \ 
\end{minted}
to
\begin{minted}[frame=lines,framesep=2mm]{bash}
DAEMON_OPTS="-a *:80 \
\end{minted}

This means that varnish will listen port 80 on webserver

Restart apache and varnish services
\begin{minted}[frame=lines,framesep=2mm]{bash}
service apache2 restart
service varnish restart
\end{minted}
Test your result using netstat command

\begin{minted}[frame=lines, framesep=2mm]{bash}
netstat -lp | grep varnish
\end{minted}


\begin{Verbatim}[frame=single,
label=Command output,framesep=2mm,rulecolor=\color{red},commandchars=\\\{\}]
margus@marguspc:~$ 
\fbox{\color{red}varnsih}  \fbox{\color{red}TODO}
\end{Verbatim}
%$


Test new system with AB utility.

Links:
\href{http://kaanon.com/blog/work/making-wordpress-shine-varnish-caching-system-part-1}{Making wordpress shine with Varnish caching system}
\href{http://kaanon.com/blog/varnish/making-wordpress-shine-varnish-caching-system-part-2}{Making wordpress shine with Varnish caching system part 2}
\href{http://www.google.com/producer/editions/CAowvZtX/full_circle_magazine_57_lite}
{Full Circle Magazine 57}

