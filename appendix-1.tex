

\chapter{Protecting Web Application Against (D)DOS Attacks}
\label{Protecting Web Application Against (D)DOS Attacks}

 
\begin{quote}
"If you tell me, I will listen. If you show me, I
will see. If you let me experience, I will learn." --- Lao Tzu (6th Century BC)
\end{quote}

\section{Introduction}
This goal of this lab is secure a web application -- WordPress against (D)DOS attacks to the level where main limitation becomes a throughput of the network. Installed and hardened server must recover after attack is ended.

Web application WordPress are used because misbehave of the default installation which can not take reasonable load.

\section{Pre-Requirements}
\begin{enumerate}
\item Preliminary GNU/Linux course~\ref{Preliminary course - dpkg based GNU/Linux} 
\item Preliminary test~\ref{Preliminary Tests} (theory and practice)
\item Knowledge about \gls{HTTP} (different request types, virtual hosts, status codes), \gls{IP} and aliases, \gls{UDP}

\end{enumerate}

If renewal is needed then following materials are suitable for rehearsal the basics of the \gls{HTTP}~\footnote{
\href{http://net.tutsplus.com/tutorials/tools-and-tips/http-the-protocol-every-web-developer-must-know-part-1/}{net.tutsplus.com - tools and tips - HTTP part 1}} \footnote{
\href{http://net.tutsplus.com/tutorials/tools-and-tips/http-the-protocol-every-web-developer-must-know-part-2/}{net.tutsplus.com - tools and tips - HTTP part 1}}

\section{Software and hardware}
Students must have possibility to run at least two virtual machines with configuration seen in table~\ref{table:HW for Wordpress lab}

\begin{table}
\centering
\caption{Hardware requirements for the (D)DOS lab}
\begin{tabular}{|c|c|c|}
\hline 
\rule[-1ex]{0pt}{2.5ex} Hardware & Server & Client \\ 
\hline 
\rule[-1ex]{0pt}{2.5ex} RAM & $>=512MB$ & $>=1GB$\\ 
\hline
\rule[-1ex]{0pt}{2.5ex} HDD & $>=8GB$ (dynamic disk) & $>=16GB$ (dynamic disk)\\ 
\hline 
\rule[-1ex]{0pt}{2.5ex} NIC 1 & NAT  & NAT \\ 
\hline 
\rule[-1ex]{0pt}{2.5ex} NIC 2 & HostOnly & HostOnly \\ 
\hline 
\rule[-1ex]{0pt}{2.5ex} OS & Ubuntu Server 12.04 LTS & Ubuntu Desktop 12.04 LTS\\ 
\hline 
\end{tabular}
\label{table:HW for Wordpress lab}
\end{table}

GNU/Linux distribution Ubuntu Server 12.04 LTS 64bit, WordPress -- latest version available, MySQL from Ubuntu repositories, Apache2 web server from repositories, GNU/Linux Ubuntu Client 12.04 LTS 64bit for performing load generation with apache2 utils.

\section{Learning Objectives}
Student installs and configures the apache2 web server and WordPress web application with \gls{MySQL} database.
Student is able to use caching technologies to protect web application against simpler DOS attacks. Student configures web service and demonstrates that can be
easily take offline using simple load generator. Minimal level: Student configures
proper caching and demonstrates that web application survives same attack.

\section{Setting up the Virtual Environment - VirtualBox sample}
Two virtual machines are needed in this lab: Server and Client.
Download server and client \gls{OVA} files from the following links:
\url{http://elab.itcollege.ee:8000/infra_klient_small.ova}
\url{http://elab.itcollege.ee:8000/infra_server.ova}

Import virtual machines (If your host computer has only 4GB RAM, then reduce client machine memory to 1GB)

Start both machines. 
{\small{If you got an error about host only network then open Main Menu, choose File Preferences and choose Network and add Host Only Network.}}

Username and password for both machines are student.

Student user are in sudo group and can start administrator shell with \emph{sudo} command.

Log on to client and add two addresses on \emph{/etc/hosts}
\begin{minted}[frame=lines,framesep=2mm]{bash}
echo "192.168.56.200 wp.planet.zz">>/etc/hosts
\end{minted}

Test \emph{wp.planet.zz} with ping command.

\section{Installation of the WordPress}
All following commands must executed as root user. To get root permissions in Ubuntu Server used in this lab type:
\begin{minted}[frame=lines,framesep=2mm]{bash}
sudo -i
\end{minted}

For installing new software, update the local package cache in client and server:
\begin{minted}[frame=lines,framesep=2mm]{bash}
apt-get update
\end{minted}

Upgrade both systems:
\begin{minted}[frame=lines,framesep=2mm]{bash}
apt-get dist-upgrade
\end{minted}

Install apache2 web server and \gls{MySQL} database on server:
\begin{minted}[frame=lines,framesep=2mm]{bash}
apt-get install apache2 mysql-server ssh php5 php5-mysql 
apt-get install apache2-utils libapache2-mod-php5
\end{minted}

Download the latest version of WordPress engine on server:
\begin{minted}[frame=lines,framesep=2mm]{bash}
wget http://wordpress.org/latest.tar.gz
\end{minted}

Unpack downloaded \emph{latest.tar.gz} archive to server's \emph{/var/www} directory using tar utility:
\begin{minted}[frame=lines,framesep=2mm]{bash}
sudo tar zxvf latest.tar.gz --directory=/var/www/
\end{minted}

On server, create new \gls{MySQL} database called \emph{wp} and database user \emph{student}. Grant all privileges on database \emph{wp} to user \emph{student}:

\begin{minted}[frame=lines, framesep=2mm]{bash}
mysql -u root -p
create database wp;
create user student;
GRANT ALL PRIVILEGES ON wp.* TO 'student'@'localhost' IDENTIFIED BY 'student';
quit
\end{minted}

On server, create a new virtual host for wordpress 

\begin{minted}[frame=lines,framesep=2mm]{bash}
cp /etc/apache2/sites-available/default /etc/apache2/sites-available/wp
\end{minted}

On server, change the owner and the group to apache2 system user/group for wordpress directories and files for ensure that web server can read and write those files.
\begin{minted}[frame=lines,framesep=2mm]{bash}
chown www-data:www-data /var/www/wordpress -R
\end{minted}

On server, change a document root directory (DocumentRoot) for new virtual-host and add ServerName field to virtualhosts configuration file \emph{/etc/apache2/sites-available/wp}


\begin{minted}[frame=lines, framesep=2mm]{bash}
ServerName	wp.planet.zz
#DocumentRoot /var/www
DocumentRoot /var/www/wordpress
\end{minted}


To enable new virtualhost for WordPress use \emph{a2ensite} utility (on server)
\begin{minted}[frame=lines,framesep=2mm]{bash}
a2ensite wp
\end{minted}

Change wordpress configuration file
\emph{/var/www/wordpress/wp-config-sample.php}

Set correct values for defines DB\_NAME, DB\_USER, DB\_PASSWORD as:
\begin{minted}[frame=lines,framesep=2mm]{php}
/** MySQL database name */
define('DB_NAME', 'wp');
/** MySQL database username */
define('DB_USER', 'student');
/** MySQL database password */
define('DB_PASSWORD', 'student');
\end{minted}

Copy sample configuration file to the real configuration file.
\begin{minted}[frame=lines,framesep=2mm]{bash}
cp  -a /var/www/wordpress/wp-config-sample.php /var/www/wordpress/wp-config.php
\end{minted}

Reload apache configuration files:
\begin{minted}[frame=lines,framesep=2mm]{bash}
service  apache2 reload
\end{minted}

Go to address http://wp.planet.zz/ using web browser.

Enter values for  Site Title, username, password and an e-mail

Choose Install

\subsection{Testing Your WordPress Installation against simpler DOS attacks}

\begin{Verbatim}[samepage=true,frame=single,
label=Discussion,framesep=2mm,rulecolor=\color{blue},commandchars=\\\{\}]
How many requests per second the default installation of WordPress will serve? 
How many parallel connections this site should handle?
How many parallel connections and requests can produce one attacker?
When the website is down? 
How many seconds client probably waits  before website considered as dead?
\end{Verbatim}

Install apache2 utils on CLIENT computer, not in the server computer.

\begin{minted}[frame=lines, framesep=2mm]{bash}
sudo apt-get update
sudo apt-get install apache2-utils
\end{minted}

In case of Fedora/CentOS/RH/Oracle Linux install httpd-utils package.

Execute Apache Benchmark program \emph{ab} with parameters discussed:
\begin{minted}[frame=lines, framesep=2mm]{bash}
ab -c<NO_CONN> -t<TIME> http://wp.planet.zz/
\end{minted}
flag c - parallel connections
flag t - time for test

\begin{minted}[frame=lines,framesep=2mm]{bash}
ab -c600 -t20 http://wp.planet.zz/
\end{minted}

In last example the ab utility makes 600 parallel connections and test takes 20 seconds.
Test results
Store test results and the command line used for tests.
Write down request per second. No of failed requests and No of completed requests.


\subsection{Hardening WordPress Installation}
After successful load generation using \emph{ab} command, the server is extremely slow and unresponsive.

\begin{Verbatim}[samepage=true,frame=single,
label=Discussion,framesep=2mm,rulecolor=\color{blue},commandchars=\\\{\}]
Why You can not login into server?
Look at the server console. What is the OOM? What is the OOM killer?
\end{Verbatim}


If needed, reboot the server. To  guarantee log in possibility into server under attack disable swap file.

Disable swap (edit /etc/fstab file or use swapoff command)


\begin{minted}[frame=lines,framesep=2mm]{bash}
swapoff -a
\end{minted}

Disable OOM killer for MySQL database. In newer kernels write -1000 to oom\_score\_adj file.

\begin{minted}[frame=lines,framesep=2mm]{bash}
echo "-1000" > /proc/$(pidof mysqld)/oom_score_adj
\end{minted}
%$
For backward compatibility with old kernels (2.6.XX series) you can use oom\_adj file
\begin{minted}[frame=lines,framesep=2mm]{bash}
echo "-17" > /proc/$(pidof mysqld)/oom_adj
\end{minted}
%$
Documentation about proc filesystem and OOM can be found from kernel.org~\footnote{kernel.org - the \emph{proc} filesystem \url{http://www.kernel.org/doc/Documentation/filesystems/proc.txt}}

Not mandatory task: Modify mysql upstart config file to set OOM adjustment score.

WordPress Supercache
Install WordPress Supercache plugin.
Change Permalinks settings
Test cache with AB

Install Varnish HTTP cache
Change apache default port to 8080
In file /etc/apache2/ports.conf
Change 80 > 8080
Like:
NameVirtualHost *:8080
Listen 8080

Or just download new file using wget 

cd /etc/apache2
mv ports.conf /root/ports.conf.old
wget http://elab.itcollege.ee:8000/Configs/apache2/ports.conf

Change all virtual hosts to use new 8080 port using text editor or sed command.

\begin{minted}[frame=lines,framesep=2mm]{bash}
sed 's/:80>/:8080>/' -i /etc/apache2/sites-enabled/wp
\end{minted}


Install varnish web accelerator 

\begin{minted}[frame=lines,framesep=2mm]{bash}
apt-get install varnish
\end{minted}

Open configuration file for varnish defaults: \emph{/etc/default/varnish} and change default listen port from 6081 to 80 \emph{varnis}  in \emph{DAEMON\_OPTS} section.
\begin{minted}[frame=lines,framesep=2mm]{bash}
DAEMON_OPTS="-a :6081 \
             -T localhost:6082 \
             -f /etc/varnish/default.vcl \
             -S /etc/varnish/secret \
             -s malloc,256m" 
\end{minted}
The port is specified with flag \emph{-a}
\begin{minted}[frame=lines,framesep=2mm]{bash}
DAEMON_OPTS="-a *:80 \
             -T localhost:6082 \
             -f /etc/varnish/default.vcl \
             -S /etc/varnish/secret \
             -s malloc,256m" 
\end{minted}

Restart apache and varnish services
\begin{minted}[frame=lines,framesep=2mm]{bash}
service apache2 restart
service varnish restart
\end{minted}
Test your result using netstat command

\begin{minted}[frame=lines, framesep=2mm]{bash}
netstat -lp | grep varnish
\end{minted}


\begin{Verbatim}[frame=single,
label=Command output,framesep=2mm,rulecolor=\color{red},commandchars=\\\{\}]
student@opiise:~$ netstat -lp | grep varnish
tcp        0      0 \fbox{\color{red}*:80}             *:*                LISTEN  1869/varnishd   
tcp        0      0 localhost:6082    *:*                LISTEN  1868/varnishd   
\end{Verbatim}
%$
%\fbox{\color{red}varnsih}  \fbox{\color{red}TODO}

Test new system with AB utility using exactly the same test parameters and conditions as before \emph{varnish}


\begin{Verbatim}[samepage=true,frame=single,
label=Discussion,framesep=2mm,rulecolor=\color{blue},commandchars=\\\{\}]
How many requests are completed during the test?
How many requests per second the hardened WordPress installation can take?
Is it now safer or attacker can take it down with same effort? 
(You can guess that something is still wrong, and figure out what exactly)
\end{Verbatim}


\begin{Verbatim}[samepage=true,frame=single,
label=Discussion,framesep=2mm,rulecolor=\color{blue},commandchars=\\\{\}]
What can be used as possible alternative for \emph{varnish} web accelerator?
What about \gls{TLS}, do You see any problems?
What about authenticated users?
\end{Verbatim}

Additional and optional reading:
\href{http://kaanon.com/blog/work/making-wordpress-shine-varnish-caching-system-part-1}{Making wordpress shine with Varnish caching system}

\href{http://kaanon.com/blog/varnish/making-wordpress-shine-varnish-caching-system-part-2}{Making wordpress shine with Varnish caching system part 2}

\href{http://www.google.com/producer/editions/CAowvZtX/full_circle_magazine_57_lite}
{Full Circle Magazine 57}

