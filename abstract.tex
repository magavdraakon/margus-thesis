\clearpage
\chapter*{Annotatsioon - Praktilise küberkaitse e-kursus
süsteemiadministraatoritele}
\label{annotatsioon}
\thispagestyle{empty}

Käesolev magistritöö käsitleb praktilise küberkaitse e-kursuse loomist IT süsteemide administreerijatele. Põhiline töös lahendatav probleem on turvateadlike IT taristu teenuste administraatorite vähsus. Probleemi lahendamiseks loodi praktiline laboritöödel põhinev kursus, mis on kasutatav nii tasemeõppes, kui ka täiendusõppes.

Kursuse koostamisel kasutati \gls{ADDIE} metoodikat, mis sobib e-kursuse loomiseks. Loodud e-kursus erineb teistest Eestis kasutusel olevatest kursustest, kuna on loodud sihtgruppi silmas pidades fookusega IT taristu teenuste kaitsele.

Suurendamaks kaitsele orienteeritud kursuse põnevust tudengite seas on kasutusel medalite süsteem ja edetabel, mis toob kaasa võistlusmomendi. Kursusel kasutatakse probleemipõhist õpet praktiliste ülesannete puhul, koostööl põhinevat õpet rühmatöö vormis laboriaruannete  koostamisel ja kogukonnapõhist õpet abistavate ja selgitavate õppematerjalide loomiseks.

Töö tulemusena valmis uus väljundipõhine kursus mahuga 6EAP-d (32h loenguid, 46h praktilist tööd, 78h iseseisvat tööd), mida piloteeriti Eesti Infotehnoloogia Kolledžis tasemeõppes ja täiendusõppes. Kursus koosneb loengutest, nende videosalvestustest, enese ja eelduste testidest, praktilistest, probleemile orienteeritud laboritöödest ja interaktiivsetest testidest.

Kursuse tagasiside on positiivne ja kursuse tulemusena paraneb tudengite ja ettevõtete süsteemide administreerijate turvateadlikus.
% Kursuse loomisel kasutati erinevate ettappide kvaliteedi hindamiseks teiste õppejõudude ja tudengite abi.

Autori roll seisnes nõuete ja õpiväljundite loomises, laboritööde ja loengumaterjalide koostamises koostöös teiste õppejõududega (panus >90\%) ja kursuse piloteerimises nii taseme, kui täiendõppes.

\clearpage
\chapter*{Annotation}
\label{annotation}
\thispagestyle{empty}

This thesis describes development of hands-on e-learning course of cyber security for IT system administrators. The main problem considered is deficiency of security aware IT system administrators. To mitigate the problem a new lab intensive, hands-on course which are usable in curriculum and also in continuous education was developed.

For course development the \gls{ADDIE} model was used because applicability for creating e-learning courses. Furthermore, the course developed differentiate from courses used in Estonia because its designed for defence of IT infrastructure and for particular target group.

To magnify adventurous of the defence oriented course the badge reward system and scoreboard was used to create competition amongst students. Moreover, a problem based learning are used for practical classes, collaboration aspects used for report writing and community based learning to create additional learning materials.

Result of the work is new outcome based e-learning course with load 6ECTS (32h lectures, 46h practical classes, 78h homework). Furthermore, the course was piloted in Estonian IT College and also in continuous education. The course contains lectures, self-tests and preliminary tests, problem oriented hands-on classes and interactive tests.

Feedback for the course was positive and security awareness has been improved amongst students and system administrators. %For ensuring a quality of the course each stage of the course was evaluated by lecturers and students.

The role of the author was establishment of the requirements and learning outcomes, authoring a learning materials together with other lecturers (>90\% by author) and piloting the course during studies and also in continuous education. 
