\chapter{Future Research}
\label{Future Research}

\lettrine[lraise=0.1, nindent=0em, slope=-.5em]{\color{Violet}M}{ost} important problems that remained unsolved are listed in no particular order: first, human aspect of the cyber defence - configuring boxes are only one aspect of cyber; second, maintaining central logging and log parsing; third, installing, and managing an \gls{IDS}/\gls{IPS} system (simpler aspects covered by \gls{WAF} and \gls{SQL} firewalls in webserver hardening lab); fourth, configuring e-mail system with antivirus, and spam filtering; fifth, authentication and authorization with Kerberos, LDAP, SAMBA4 domain; sixth, the IPv6 is mandatory for newer infrastructure and this needs to be implemented in each lab.


The development process of curriculum ends when curriculum itself is obsolete. Therefore additional seminars with partners, other educational institutions and private companies will be carried out. For example, the next phase will be collecting information about Locked Shield 2013 and needed skillset for technicians, and also other aspects that system administrators should address.

Although it seems that more should be done in the future compared to the work done, all aspects can not be implemented during one e-learning course with the load of 6 \gls{ECTS}. New areas of possible e-learning courses are listed in Appendix~\ref{Appendix:Lab proposals for the future} on page~\pageref{Appendix:Lab proposals for the future}. In order to support new scoring system of implementing badge reward system, the virtual laboratory system will be redesigned in summer 2013.


The IT operations can protect the systems but if the system itself is weak it leads to another problem -- lack of security aware software developers. In the author’s opinion this the challenge that will emerge soon.
