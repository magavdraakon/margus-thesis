\annotation{Annotation}

Cyber security field is rapidly growing and need for highly educated and security aware ITC specialist is increasing. Due to increased malicious activities in the Internet the education in ITC field should emphasize knowledge and skills in cyber security field.
The Estonian IT College (EITC) is focused on applied higher education. Study programs in EITC include development, administration and system analysis. Study program focuses on practical learning by doing approach to ITC subjects. The students gain skills and knowledge as they progress through curricula lectures and practical classes. The studies include lectures, practical classes and independent work as homework. By and large, one subject is divided as follows: 25\% lectures, 25\% practical classes and 50\% homework which is mostly working with materials (books, web articles). In the worst case, practical work constitutes only 25\% and takes place at EITC computer classes. Students are not interested in learning mere theory. The formulas are not seen necessary nor linked to their study area or future job. Theory that is not used will be forgotten quickly. In a few years students won't even remember if a specific topic was covered or not. Applied education should introduce practical approach and learning by doing. In ITC field practical classes and practical homework is the key to achieving acceptable results. Students appreciate learning IT system administration by building complex systems using virtual distance laboratory environment and gaining practical experience during this process following learning-by-doing study model.
The problem with practical classes in the cyber field that the practical approach requires hardware, software and preconfigured laboratory environment. However, the  distance laboratory environment can be developed in a manner  that practical classes can be accessed from any Internet connected computer using web browser. Moreover, using virtual and game-like environments is a contemporary approach for teaching IT System administration and programming focusing to the cyber security requirements. The outcome of the developed environment and contemporary teaching methods are students with increased motivation and skills in the cyber field. The tasks given to the students for solving have to be associated with the practical applicability of the theoretical knowledge.

Developed systems allows to  increase of the proportion of practical work by converting independent and theoretical homework to  the practical classes. The quality of studies will improve due to increased amount of practical hands-on classes. The developed system uses only open source components and released using open source compatible MIT licence. The development and hardware are funded by European Social Fund (ESF). We expect to have results of using virtual laboratory system to be available in 2013.


