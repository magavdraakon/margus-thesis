\chapter{Conclusions}
\label{conclusion}
\lettrine[lraise=0.1, nindent=0em, slope=-.5em]{\color{Violet}L}{ack} of qualified and security aware system administrators in the region is a problem that hinders the growth of e-services and accumulates risks. Therefore the goal of this thesis was to develop a practical e-learning course for system administrators. The new course is also applicable for \gls{EITC} students and for distance learners. 

In cooperation with \gls{EITC} partners, the IT system administration curriculum was analysed and plans for a new e-learning course were made. Therefore, the author of this thesis established requirements, instructional goals, learning outcomes, learning objectives and course content using the \gls{ADDIE} model as instructional design framework.

The problem discussed in the thesis is not novel in essence and it has been addressed in universities and private training companies. Therefore the field is well covered with similar research. However, the current thesis applies instructional design method and local requirements to design new defence oriented e-learning course. Moreover, by combining reward badge system as a competitive moment with distance laboratory environment, the result is the e-learning course designed for this particular target group to achieve established goals in a playful and intensive way.

The pilot labs were held with different groups such as foreign system administrators, foreign students, Estonian system administrators and students. The course feedback has been positive and proves that chosen method was suitable for developing the e-learning course.


In conclusion, the \gls{EITC} has a new course in IT System Administration curriculum and more than 80 system administrators will be educated every year. Consequently,  there will be more security aware system administrators with proper knowledge and skills in Estonia.
