\chapter{Conclusion}
\label{conclusion}
\lettrine[lraise=0.1, nindent=0em, slope=-.5em]{\color{Violet}T}{he} lack of qualified and security aware system administrators on region is the problem that suppresses grow or e-services and accumulates risks. Therefore the goal of this thesis was to develop a practical e-learning course for system administrators. Moreover, new course is applicable for \gls{EITC} students and also for distance learners. 

In cooperation with \gls{EITC} partners, the IT system administration curriculum was analysed and plans for new e-learning course were created. Therefore, author of this thesis established requirements, instructional goals, learning outcomes, learning objectives and course content using the \gls{ADDIE} model as instructional design framework. 

Likewise in this thesis the hands-on security labs are and not new and are commonly used and taught by several universities and private training companies. Therefore the field is well covered with similar research. However, the current thesis applies instructional design and local requirements for to design new defence oriented e-learning course. Moreover, combining reward badge system as competition moment with distance laboratory environment, the result is the e-learning course designed for this particular target group to achieve established goals in playful and intensive way.

The pilot labs are being held with different groups as foreign system administrators, foreign students, Estonian system administrators and students and the course feedback was positive and proves that chosen method was suitable for developing the e-learning course.

Even the IT system administration curriculum is corresponds better to the industry requirements the IT system development curriculum is not covered and lefts big field waiting for futures action.

In conclusion, the \gls{EITC} have new course in IT System Administration curriculum and more then 80 system administrators will be educated in each year. Therefore the Estonia receives more security aware system administrators with proper knowledge and skills.
