\chapter{Letter from CERT.EE to Rector of Esonian IT College}
\label{Letter from CERT.EE to Rector of Esonian IT College}
\begin{spacing}{1}
\small
%\begin{alltt}
---Original Message-----

From: CERT.EE töötaja\par
Sent: Tuesday, February 10, 2009 3:55 PM \par
To: Eesti Infotehnoloogia Kolledž Rektor \par
Cc: ***** \par
Subject: Täiendkoolitus\par

Tere
Meil on üks probleem  :
Pole piisavalt haritud administraatoreid omavalitsustes ja 
muudes riigiasutustes ning väikese ja keskmise suurusega organisatsioonides.

Probleem jaguneb mitmeks väiksemaks alam probellmiks :
enamik administraatoreid on nn iseõppijad (mis on muidugi tore!)
ja omandanud [hädapärased] teadmised ja kogemused töö käigus
kutse ja rakenduskõrghariduse raames ei anta õpilastele juur- ja
 turvateenuste alal vajaliku põhjalikusega teadmisi 
(vahest on liiga vara spetsialiseeruda ?!)

täiendkoolituse turul olemas vaid tootjate endi tarkvaratoodete põhised kursused 
(tihti kontori tarkvara üldkursused)

tegelik üldine teadmiste ja oskuste tase sihtrühmas 
ei ole piisav hästi toimiva süsteemi haldamiseks 
ega tõrgete kõrvaldamiseks
Täiendkoolituse turul puuduvad nimetatud sihtrühmale 
vajalikud kursused.

Nii võiks välja näha kohaliku omavalitsuse itimehe ja ülemuse arenguvestluse üks osa:
itimees: Tahan minna nädalaks koolitusele, maksab 15 tuhat, see teeb vaid 3 tuhat ühe päeva eest. ülemus ütleb: Oota, mõtleme, aga nädalaks sind ära lasta ei saa ja kallis on see ka, ikkagi 15 tuhat ülemus mõtleb: .oO(saadad koolitusele, ja pärast läheb teise kohta suure palga peale, las parem sekretär käib wördi koolitusel ära)

Lahendus :
Valitsus (?HM, MKM, KM?), veel parem EU maksab keskelt kinni kursuste ettevalmistamise ja 3 aasta jooksul sihtrühma koolituse plaan sihtrühma koolituseks :

a) ette valmistada nädalased kursused (40 x 45 min) järgnevatel teemadel

!) kõik kursused OpenBSD baasil, kuna *BSD perekond on laialt levinud platform juurteenuste jaoks ja võimalik saadud teadmisi ja oskusi rakenda laiemalt kui vaid ühe tootja/tarkvara puhul \emph{mida oligi vaja} ;)

\begin{itemize}
	\item[0)] sissejuhatus: IPv4/IPv6, TCP/IP, kahendarvutused.... (anda alused järgmistele kursustele)
	\item[1)] tulemüüri ülespanek, seadistamine ja igapäevane haldus
	\item[2)] aja- ja nimeteenuse ülespanek, seadistamine ja igapäevane haldus
	\item[3)] veebiteenuse ülespanek, seadistamine ja igapäevane haldus
	\item[4)] postiteenuse ülespanek, seadistamine ja igapäevane haldus
	\item[5)] logihalduse ülespanek, seadistamine ja igapäevane haldus
	\item[6)] ründetuvastus ja intsidentide halduse süsteemi ülespanek, seadistamine ja igapäevane haldus
	\item[7)] Loov probleemi lahendus ja haldus.

\end{itemize}


b) viia koolitusi läbi kahe aasta jooksul

TULEMUS:
suurem enamus väikese ja keskmise suurusega organistasioonide juurteenuste administraatoreid oskab oma tööd heal või keskmisel tasemel.


Lahendusele me oleme leidnud mõningad allikad mis eeldavad kutse või kõrgharidusega tegeleva asutuse kaasamist või isegi talle projektis vedava rolli andmist.
Hea meelega saaks teiega järgmisel nädala esimesel poolel teiega kokku ja räägiks meie poolsest nägemusest lahendustele ning kuulaks teie poolset arvamust idee räideviimise võimaluste kohta .

Lugupidamsiega ....
%\end{alltt}
\end{spacing}
\normalfont
