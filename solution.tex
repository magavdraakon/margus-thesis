\chapter{Solution}
\label{solution}
\lettrine[lraise=0.1, nindent=0em, slope=-.5em]{\color{Violet}T}{he} solution chapter is divided into four parts. First, the design section draws on information gathered during the analysis phase of the  \gls{ADDIE} process. The main goals are identifying the learning objectives, composing tests, choosing course format, and planning the learning process~\citep{website:design_phase_ADDIE}.  Second, the technical implementation of the e-learning course is usually a sub-part of the previous part, the design process of the~\gls{ADDIE}  model. However, the section focusing on the technical considerations deserves a separate part in the context of this e-learning course because its importance and  volume. Third, the section dedicated to the development phase describes authoring of the learning material. The fourth section expands upon the implementation phase, covering piloting of the course, where each part of the course is a run-through by certain students/trainers to get feedback about timing and consistency of content~\citep{website:design_phase_ADDIE}.



 
\section{Designing and Planning the learning process}
The Design phase contains four steps: first, designing the learning objectives; second, designing the assessments for each objective; third, choosing the course format; fourth, creating an instructional strategy~\citep{website:design_phase_ADDIE}.  However, in this thesis, the assessment tests are designed together with the learning objectives to simplify reading the topics.

\subsection{Design of the course content and learning objectives}

In the analysis phase the learning outcomes were established. Two components need to be considered when designing adequate course content -- learning objectives and assessment tests. Similarly to the software development field the tests should be designed first to ensure creation of consistent learning materials and labs. Therefore all learning objectives are presented along with evaluation methods and values.

The terms and topics used in assessment should be kept on the level the students should know at the end of the lab and derived from learning objectives established in analysis phase.

Developed tests and materials of the course will be listed in course syllabus described in point ~\ref{Course Syllabus}.

\subsubsection{Pre-requirement courses}

During the analysis of the target group the need for preliminary course emerged to homogenize the knowledge and skill level of the participants.

The objectives and assessments for preliminary course are the following.
\begin{itemize}
\item Students are able to work with command line using GNU/Linux, work with files, manage software, manage disks and partitions, manage users and groups, configure networks and user's login session.  The pass mark is more than 50\% of randomly chosen test version, referred to in   Appendix~\ref{Preliminary Tests} in Table~\ref{tab:preliminary_practical_test}.
\item Students are able to explain basic terminology of operating systems such as kernel, GUI, shell, Virtual memory, authentication, authorization, RAM, cache, buffer, latency, throughput, file system, process, thread, password hash, DAC, MAC, RBAC, command parameter, command flag, file system hierarchy, environment variable). The pass mark is over 51\% of closed book test such as presented in a sample in  Appendix~\ref{Preliminary Tests}.
\item Students are able to configure a network of the GNU/Linux and explain terms
such as gateway, netmask, IP address, port, IP alias, DNS servers. Minimal level
to pass is successfully configuring network for lab machines.
\item Students are able to read/modify and create simpler BASH, Python and PowerShell scripts. Pass mark stands at creating scripts in all common language constructions at least in one scripting language. Powershell scripting is included because the following labs also contain integration labs with Active Directory, SAMBA4, GNU/Linux servers and workstations and Windows workstations.
\end{itemize}

Deriving from previous objectives, six practical classes are needed:~\footnote{All times are given in academic hours or days which equal 8 academic hours.}

\begin{enumerate}[label=LAB \arabic*.,leftmargin=*]
\item Operating system basics (one day)
\item Basic networking IPv4/IPv6, TCP/IP (one day)
\item GNU/Linux basics (and OpenBSD/FreeBSD basics) (16h) as described in Appendix~\ref{Preliminary course - dpkg based GNU/Linux} on page~\pageref{Preliminary course - dpkg based GNU/Linux}
\item Scripting in BASH (2 days)
\item Scripting in Python (1.5 days)
\item Scripting in PowerShell (1.5 days)
\end{enumerate}
After implementing the course exact load is known numbers will be arranged.

\subsubsection{Root Services}
In this particular case the term "root services" is defined  as \gls{NTP}, \gls{DNS}, \gls{DHCP} services.
After finishing this lab block, the students are able to configure root services and use those services in following labs.
The objectives and assessments for Root Services lab are:
\begin{itemize}
\item After finishing this lab, student is able to install \gls{NTP} service on the server and on the client computer and configure client to use internal server (pool) and server to use upstream \gls{NTP} service and fall-back services. Minimal level to pass is achieved if services are configured, and student demonstrates debug skills with different tools and explains basic terms and (pool, stratum, delay , offset, jitter, drift)
\item After finishing this lab, student is able to install \gls{DNS} service and configure clients for new server. Minimally, students are able to configure zones, reverse zones, master -- slave replica, forwarding, different type of records (such as MX, A, CNAME, TXT for SPF, PTR) and use basic management utilities to do following tasks: reload zone, flush name, flush cache, add records dynamically, freeze and thaw zones). Configured service should able to do make recursive queries for one particular subnet and student are able to explain which \gls{DNS} attacks are common in Internet and what  an Open Resolver is. Student tests the nameserver of the \gls{EITC} and explains what is wrong with that.
\item After finishing this lab, student is able to install a \gls{DHCP} server and configure hosts using this service. Minimal level to pass is the following: student installs and configures a service that gives networking configuration to client machine. Service updates \gls{DNS} records using shared key (Mandatory Access Control must not disabled for pass).
\end{itemize}
Deriving from previous objectives, three practical classes are needed:
\begin{enumerate}[label=LAB \arabic*.,leftmargin=*]
  	\item NTP (4h)
  	\item DNS (2.5 days)
  	\item DHCP (one day)
\end{enumerate}

\subsubsection{Web and File services}
Configuring and securing web servers is an essential skill required from system administrators. Therefore, web server installation, configuration and hardening are covered in this block.

The learning objectives and assessments for Web and File services are the following:
\begin{itemize}
\item After completing the web and file services block student is able to install web server and web application with database and several virtual hosts and configure \gls{TLS}. Minimal level for passing is reached when a web server with two virtual hosts accepting \gls{HTTP} and \gls{HTTPS} connections is installed and IP aliases for  \gls{HTTPS} and/or \gls{SNI} have been configured.
\item Student is able to use caching technologies to protect web application against simpler \gls{DOS} attacks. Student configures web service and demonstrates that installed application can be easily taken offline using a simple load generator. Then, the student configures web application accelerator as mitigation method against the  \gls{DOS} attack.  Minimal level is reached if the student configures proper caching and demonstrates that web application survives \gls{DOS} attack.
\item Student is able to install different application firewalls such as \gls{SQL} firewall and web application firewall. Minimal level is reached if the student demonstrates that different types of attacks are possible and successful against the vulnerable web application, installs \gls{SQL} firewall and demonstrates that basic \gls{SQLi} attacks are blocked, demonstrates that several web application attacks are still possible after installing the \gls{SQL} firewall such as reflected \gls{XSS} and stored \gls{XSS}, command injection and \gls{CSRF}, installs application firewall before web application and demonstrates that previously succeeded attacks (at least \gls{XSS}) are stopped.
\item Student is capable of installing a file server and configure shares, permissions, groups. The passing criteria are: student installs the service and configures two shares and group based permissions for each share, configures client machine to mount one share when user logs on and other share should mounted on boot.\end{itemize}


Deriving from previous objectives, four practical classes are needed:
\begin{enumerate}[label=LAB \arabic*.,leftmargin=*]
   \item Web server basics - installation and configuring web server (4h)
  	\item Web server security - Protecting Web Application Against (D)DOS Attacks (6h)
  	\item Web server security - securing a vulnerable web application by using application firewalls (6h)
  	\item Fileserver installation and configuration (4h)
\end{enumerate}


\subsection{Choosing the course format}

When developing e-learning courses, choosing the course format is a quick decision but one that strongly influences all participants~\citep[p.14]{OppeArenduskeskus2010}. 

The common course delivery formats used in e-learning and blended learning (combined form of instruction-led and e-learning) are: 
\begin{itemize}
\item Asynchronous e-learning - \gls{SIS}, wikis, \gls{LMS}'s, forums, blogs and other collaboration systems;
\item Synchronous e-learning - virtual distance laboratories, remote access for some servers, Skype and other online communications;
\item Instructor-led lessons - practical classes;
\item Self-study - homework, reading books and other resources.
\end{itemize}

For this course several of these methods are used in different cases. First, self-study is made possible because all the materials and virtual machines are available from web (except some video materials). Second, a combination of self-study and instruction-led lessons are used for continuous education groups. Third, a asynchronous e-learning is commonly used and is combined with instruction-led sessions.

For this e-learning course the students can choose a combination of asynchronous e-learning, instruction-led lessons (maximum half of the course) or self-study using lecture recordings.

The student collaboration is preferred when preparing documentation in wiki format, reviewing others work and also grading others work because this method motivates students to produce better documentations.




\subsection{Instructional strategy}
The instructional strategy for courses focuses on guaranteeing learning outcomes by motivating students, using proper content and activities planning~\citep{website:design_phase_ADDIE}.


\subsubsection{Pre-instructional activities}
The main goal of pre-instructional activities is to motivate the students by explaining why started topic must be covered and what the student will be able to do after practical classes. Thereafter, learning objectives, assessment requirements and pre-requirements for the class are presented. The minimal requirement level is presented while providing motivated students with higher challenges.


\subsubsection{Content Presentation}
Most today’s students are not interested in receiving only theoretical lectures. For example, 1/2 of students do not have sufficient knowledge to fully understand the topic, 1/4 of the students already know the given aspects and only 1/4 of the students are on the suitable level. Although, the exact numbers vary, it is possible that a similar situation occurs often. Therefore, practical classes and lectures are combined and all taking place in computer classes. Every student should take 15-30 minute long theoretical blocks followed by practical activities. In case of e-learning students can browse video recordings and do their labs when and where it is suitable for them. The content presentation should contain a principle the author would like to call a \emph{Command Dojo}, where all the  participants are doing the same exercise with the help of the lectures as masters in the classroom or by using screen-cast. The name of the method is derived from \gls{Coding Dojo} in software development~\footnote{A Coding Dojo is a meeting where a bunch of coders get together to work on a programming challenge \url{http://codingdojo.org/cgi-bin/wiki.pl?WhatIsCodingDojo}}.
The reason behind this method is simple: Students need a good guide to follow and learn from, but to demonstrate learned skills the students get a new goal where they need to configure services without blindly coping and pasting from lab materials.

\subsubsection{Learner's feedback and assessment}
After every class, the lecturer should gather feedback on objectives, theoretical parts, labs and assessments because the course is very intensive. When more then 1/3 students can not follow due to some problems then they will fail in the next block because they are linked by topic.
It is also possible that asking students how they feel about the course provides valuable feedback.

\subsubsection{Follow-through activities}
Active discussions are needed to explain some situations. Possible discussion topics are shown in the lab materials as questions for the students that are highlighted in a blue box with the caption \emph{discussion}. 

\begin{Verbatim}[frame=single,
label=Discussion,framesep=2mm,rulecolor=\color{blue},commandchars=\\\{\}]
Why You can not login into server?
Look at the server console. What is the OOM? What is the OOM killer?
\end{Verbatim}

In case of distance study and self-study the discussions should be held using the course Skype list, because it is suitable for group discussions and has been  successfully tested on bigger courses~\citep{website:kakk_teistmoodi_e}. To conclude the discussion the lecturer must give feedback on each discussion topic using the same channel or course e-mail list.


\subsection{Pedagogical view of the e-course}
Different Pedagogical strategies can be used during the learning process such as \citep{OppeArenduskeskus2010}:

\begin{enumerate}
\item problem based learning -- demands an analytical approach from the student by solving cases based on scenarios derived from real situations;
\item collaboration based learning -- based on group-work and cooperation;
\item community based learning -- collaboration is community based, helps students to learn from each-other.
\end{enumerate}

All three aspects are used and combined in this course. The problem based learning method is used in labs. The collaboration learning is used for documenting a installed service using the course wiki that aggregates this information. Community based learning is used while reviewing the wiki articles. 

The problem based learning is used in case of continuous education because of its intensive nature and time limitations that exclude the possibility to give home assignments.

\subsection{Planning grading/assessment techniques}
The assessment is used for two proposes: Firstly, to ensure the achievement of the learning outcomes. Secondly, as a form of feedback for the student. By planning assessment for an e-learning course the common choices are the following: self-assessment, computer aided assessment, tutor assessment and peer assessment. ~\citep{OppeArenduskeskus2010}

Self-assessment is used as self tests before the course and for  deciding the need for a preliminary course.

For example, if a system administrator wants to enter the Web and File service course a self-test has to be done to ensure the presence of knowledge needed in the course.

Although, the computer assessment is easier and less time consuming compared to tutor assessment it is insufficient to guarantee the learning outcomes of the student. The computer assessment is only used for guiding and giving feedback to the participant. For example: In case of the insecure web application lab the student can execute a script that tests the applications vulnerabilities by performing \gls{SQLi} attacks and testing if the attack is filtered by the \gls{SQL} firewall or not. The testing script itself is available to the user and in case of this particular script, was written by other students as a homework in a preliminary scripting course.


The tutor assessment is used to grade the students lab performance and is time consuming because every student defends their lab solution by reconfiguring services and explaining the architecture and configuration of the solution.

Peer assessment is used in case of grading the documentation. The grade points are given by peer students and tutor assessment is used to grade the graders.

Proper grading is one way to motivate students. Moreover, a competition moment while performing labs seen by other students gives extra motivation to skilled students but may demotivate weaker ones~\citep{KasakKaur}. In the authors opinion the competition moment combined with the offensive \gls{CTF} type course gives a motivation impulse to most of the students. Although, this course focuses on defending IT systems, the motivation problem is still not solved but one possible solution is in implementation.

The idea for a motivation system for the new course is to implement a scoreboard and a reward system based on completed and graded labs. For example: when the student secures an \emph{apache} web server using mod security rules the badge is added to the student's profile in the lab system. A steel coloured badge with the \emph{apache} icon is given for using a application firewall and protecting the system. A silver coloured shield badge is given when a proper report is submitted to "authorities" describing when, what happened and what the student did for the mitigation. A golden badge is given for writing new rules, new scripts or new log parsers to help the administrator to deal with the problem. The technical  realization of this system will be a future work in the next iteration of the course.

According to the \gls{ADDIE} model the next step should be "Choosing technological tools" as audio/video programs and \gls{LMS} system choices, file formats, media authoring programs and choice of collaboration environments~\citep{OppeArenduskeskus2010}. However, this course needs more detailed technical choices to cover learning objectives and needs for a virtualized environment. Therefore, the choices are described in a separate section~\ref{Technical implementation of the e-learning course}.

\section{Technical implementation of the e-learning course}
\label{Technical implementation of the e-learning course}

The general aspects for consideration when choosing technical tools and systems for a e-learning course are the following: availability of the e-learning course, usability of the e-learning course, student motivation, adaptivity, suitability for collaboration, standard compliance~\citep{OppeArenduskeskus2010}:

Technological tools are needed for following: 
\begin{itemize}
\item tools for sharing study materials;
\item tools for collaboration and communication between students and lectures;
\item tools for implementing virtualization environments;
\item Tools for implementing lab scenarios such as: web applications, services, testing programs.
\end{itemize}


For sharing study materials the \gls{EITC} wiki~\footnote{\gls{EITC} wiki - \url{https://wiki.itcollege.ee/}} is used because it is publicly accessible and students can add and edit materials. Students can grade other students works allowing them to share their knowledge. Study materials are available in MediaWiki, pdf, OpenDocument and text formats. Some tests and temporary assignments are stored into google docs and made available by using shared links.

The wiki is used by the students to collaborate on assignments and to give feedback to others submissions.
All homework and reviews are publicly accessible. Even the question asked during the lab defence are published in the wiki by students, helping each other to prepare for future defences.

E-mail and Skype are being used for daily communication and a course forum is to be implemented in the future. 


Virtualization environments are used for all labs. Several virtualization solutions can be used in this e-learning course. For availability the virtual machines are distributed as Open Virtualization Format (\gls{OVA}) files that are usable in different environments~\footnote{ Distributed Management Task Force (\gls{DMTF}) OVF - \url{http://www.dmtf.org/standards/ovf}}.


Remote access to the \gls{EITC} computer class environment should be provided to ensure availability.
Therefore, an environment of distance study is implemented  and constantly developed in \gls{EITC} to accept the  needs of new e-learning courses. The development of this course initiated new developments in the distance study environment.
The author of this thesis is an architect of the distance study environment and is also its back- end programmer. 

A brief overview of the system is given in the next section.

\subsection{The Environment of Distance Study}
\label{The Environment of Distance Study}
The motivation to develop the environment of distance study was initiated by the need to increase the amount of practical hands-on work in \gls{EITC}. However, the virtualization used in computer classes to teach system administration subjects has limitations. The students were only able to work on labs during scheduled labs.
The students virtual machines are stored on local disks of the computer making changing computers impractical. Some students preferred to use their own laptops for running virtual machines. However, the hardware might not be sufficient to support bigger labs with several virtual machines.

Therefore, the development of the environment of distance study was initiated.

The environment allows students to start virtual labs with pre-configured virtual machines.
 
\subsubsection{Technical implementation}
For virtualization \gls{API} the libvirt is being used because of support for common virtualization hypervisors like \gls{KVM}, Xen, VmWare ESX, LXC and others~\footnote{libvirt - The virtualization API \url{http://libvirt.org/}}. The web interface is being developed using the Ruby on Rails framework. \gls{LDAP} is being used for authentication and Ubuntu Server 12.04 LTS 64bit as the host operating system. The architecture of the distance lab system is shown in Figure~\ref{fig:Architecture of Distance Laboratory}.

\begin{figure}[ht]
\centering
\includegraphics[width=0.8\textwidth]{architecture.png}
\caption{Architecture of Distance Laboratory}
\label{fig:Architecture of Distance Laboratory}
\end{figure}

To support new ideas from this thesis the following development has to be done:
\begin{itemize}
\item implementing a scoring table with virtual reward badges;
\item implementing a new  network configuration infrastructure to support several internal networks isolating \gls{DOS} attacks generated by the students and the \gls{DHCP} traffic;
\item implementing a automatic feedback system for students achievements  in defensive labs.
\end{itemize}

The development is not finished except for the network configuration part. Most of the programming work was done by students as diploma or master projects and supervised by author of this thesis.

The source code of the distance laboratory system is publicly available in a \gls{git} repository~\footnote{The distance laboratory system i-tee -- \url{https://github.com/magavdraakon/i-tee}}.
The system itself is accessible using a \gls{EITC} account~\footnote{The i-tee distance laboratory system -- \url{https://elab.itcollege.ee/}}.
%\subsubsection{New developments for distance study environment}
%
%\begin{itemize}
%	\item normal traffic generator
%	\item malicious traffic generator
%	\item availability monitor (for grading)
%\end{itemize}

\subsection{Operating Systems used in labs}

According to the requirement established in the analysis, the labs should use open source operating systems.
Several operating systems are being used in labs to maintain diversity and avoid vendor locking.
Lab materials are developed with Ubuntu LTS in mind, because of its support and popularity. To get graded the student should choose one different system for defence in at least one lab. The system could be chosen from the following list: OpenBSD, FreeBSD, OpenSolaris, Debian GNU/Linux, Fedora, CentOS, Oracle Linux, OpenSuse. The only restriction is that the chosen system should use a different packaging. For example if a student chooses Ubuntu to do the \gls{DNS} lab then for the \gls{DHCP} lab a operating system without debian packaging system should be used.

Labs with offensive parts are done using Kali GNU/Linux distribution but students can also choose BackTrack.

The main reason the choice is left open is that in labs each student must demonstrate skills and knowledge according to the learning objectives that are more important than knowledge of one system.

\subsection{Choosing software for Root Services lab}
The Root Services labs contains \gls{NTP}, \gls{DNS} and \gls{DHCP} services. However, for fulfilling the learning objectives the software should work on the chosen lab platform -- GNU/Linux Ubuntu Server. 

\subsubsection{The Network Time Protocol server}
Possible choices for \gls{NTP} server software are \emph{OpenNTPD}~\footnote{OpenNTPD \url{http://www.openntpd.org/}
} from OpenBSD  project and the Network Time Protocol Distribution \emph{ntpd} from  Internet Systems Consortium \gls{ISC}~\footnote{Network Time Protocol Distribution \url{http://support.ntp.org/}}. Both packages are installable from ubuntu repositories using the \emph{apt-get} command. The Network Time Protocol Distribution are stored in the main ubuntu repository but OpenNTPD is from the universe section. Therefore the \gls{ntpd} is  slightly more supported by Ubuntu developers.
However, the OpenNTP is designed to be a free, simple and secure implementation of the \gls{NTP} protocol. The \gls{ISC}'s \emph{ntpd} software is free, \gls{IETF} standard compliant and from main/net repository~\footnote{Information from \emph{apt-get show ntp} and \emph{apt-get show openntpd}}. Therefore the \emph{ntpd} daemon was chosen for \gls{NTP} lab.
\subsubsection{The Domain Name System server}
Several \gls{DNS} servers can be used in the lab such as: MaraDNS~\footnote{ MaraDNS is open source, lightweight \gls{DNS} server -- \url{http://www.maradns.org/}}, PowerDNS~\footnote{PowerDNS an open source feature rich \gls{DNS} server -- \url{https://www.powerdns.com/}}, Unbound~\footnote{Unbound is a validating, recursive, and caching \gls{DNS} resolver -- \url{http://unbound.net/}}, NSD~\footnote{NSD is an authoritative only, high performance, simple and open source name server - \url{http://www.nlnetlabs.nl/projects/nsd/}} and \gls{BIND}, because of installation can be done using standard packages from Ubuntu GNU/Linux repositories\footnote{Ubuntu packages --  \url{http://packages.ubuntu.com/}}.

Several \gls{DNS} implementations are not considered for lab, because they lack support for recursive queries or  were designed to be caching only name servers. However, the current \gls{DNS} lab is not using the \gls{DNSSEC} standard, its support is needed for future improvements. Therefore the unbound, NSD and \gls{BIND} are possible choices that are installable from the Ubuntu package repositories. The NSD is suitable for building authoritative (only) servers and can not be used in a \gls{DNS} lab alone.
However, the unbound with NSD or \gls{BIND} alone is suitable for this lab. The \gls{BIND} name-server has a wider user base and a number of installations~\footnote{The \gls{ISC} \gls{BIND} -- \url{https://www.isc.org/software/bind}}. Therefore, the \gls{BIND} name server was chosen for this lab.


\subsection{Choosing software for lab: Protecting Web Application Against (D)DOS Attacks}

\subsubsection{Web server}
According to Netcraft Web Server Survey (May 2013) the web server shares of active websites are: Apache~\footnote{Apache \gls{HTTP} server -- \url{http://projects.apache.org/projects/http_server.html}} -- 55.07\%,  nginx~\footnote{nginx is an HTTP and reverse proxy server -- \url{http://nginx.org/en/}} -- 13.27\%	, Microsoft Internet Information Server~\footnote{Microsoft Internet Information server -- http://www.iis.net/} -- 11.08\% \citep{website:netcraft_web}. Microsoft IIS does not qualify because it is not open source software.  Therefore, Apache is chosen as the primary web server for the lab because of its market share. However, the NGINX is also a important platform and will be used for \gls{TLS} termination in the lab, because students should be able to configure \gls{HTTPS} as well.

\subsubsection{Caching web application acceleration server}
Web acceleration servers are used to reduce the load on the web server and as a mitigation method for small grade denial of service attacks. Is is a possible solution in case of a small attack traffic, but it is useless when all network capacity is occupied by the attack. 
Several popular web application acceleration and caching servers are: Varnish Cache~\footnote{Varnish is a web application accelerator -- \url{https://www.varnish-cache.org}}, NGINX~\footnote{Nginx is an HTTP and reverse proxy server -- \url{http://nginx.org/en/}}, Squid~\footnote{Squid is a caching proxy for the Web -- \url{http://www.squid-cache.org/}}.
The personal cache systems, non open source systems and hardware acceleration systems are not compared because of license or technical limitations.
Even though, the Squid and Nginx are usable for web application acceleration, the Varnish Cache has its own configuration language, giving it an advantage for custom filtering. Therefore, the Varnish Cache was chosen for this lab.

\subsubsection{Web application for testing the web acceleration}
The list of different web applications is long and makes it difficult to give a reasonable choice.
Some applications are common and suitable for load testing and web acceleration for example: WordPress~\footnote{WordPress is open souce website or blog engine -- \url{http://wordpress.org/}}, MediaWiki~\footnote{MediaWiki is open source wiki package -- \url{http://www.mediawiki.org/wiki/MediaWiki}} and  Drupal~\footnote{Drupal is an open source content management platform -- \url{http://drupal.org/}}. Even though, the students can choose their own web application from the previous list, the lab guide covers WordPress because of its easy installation and good documentation.


\subsection{Choosing a vulnerable web application for Protecting an Insecure Web Application lab}

The main need for a vulnerable web application comes from the scenario: Each student installs a vulnerable system and must stop basic attacks without reprogramming a web application.

Although, ready made virtual appliances can be used to install a vulnerable web application the
system administrator should be able to install it himself, helping him to understand the main architecture of a web application to choose suitable protection methods. Therefore, the chosen application should be free and open source, easily installable, implementing  at least stored and reflected \gls{XSS}, several injection type attacks like \gls{SQLi} (usual and blind) and \gls{CSRF}.

\subsubsection{WebGoat}
WebGoat is a free, open source insecure J2EE web application designed to teach web application security lessons~\footnote{WebGoat -- \url{https://www.owasp.org/index.php/Category:OWASP_WebGoat_Project}}.  Even though, the WebGoat is one of the best applications for teaching web vulnerabilities, the installation and J2EE requirement is not suitable for system administration students because it has too many steps to follow.


\subsubsection{Damn Vulnerable Web Application}
The Damn Vulnerable Web Application \gls{DVWA} is web application with several vulnerabilities. It is suitable for testing several security vulnerabilities and tools. Even though, the tool does not implement all \gls{OWASP} top ten attacks, the most relevant are presented. The tool is written using PHP/MySQL which are taught to all \gls{EITC} students and usually known by system administrators. The tool is designed for learning and students can choose the difficulty level of exploiting~\citep{website:dvwa}.

The program is easy to install. It has integrated study materials and variable levels of vulnerabilities.
The vulnerability coverage is not best compared to WebGoat but is sufficient for this lab.

\subsubsection{NOWASP (Mutillidae)}
NOWASP (Mutillidae)~\footnote{NOWASP (Mutillidae) -- \url{http://sourceforge.net/projects/mutillidae/}} web pen-test practice application is a free, open source vulnerable web-application for labs, security enthusiasts, classrooms, \gls{CTF}, and vulnerability assessment tool targets. \citep{website:Mutillidae} Although, the tool documentation has videos, study materials and a good support for \gls{OWASP} top ten vulnerabilities, but the development relies on one person and the community support is thin.
\subsubsection{SQLol}
SQLol~\footnote{SQLol -- \url{https://github.com/SpiderLabs/SQLol
}} is a free and open source web application designed to test \gls{SQLi} type attacks and is compatible with \gls{MySQL}, gls{PostgreSQL}. Even though, the application has comprehensive capabilities on \gls{SQLi}, the other vulnerabilities are not covered.


In conclusion, because of the defensive nature of the developed course, only simple vulnerabilities are needed to demonstrate a problem and all vulnerable applications are suitable for the lab. To create diversity all the applications that can be installed easily may be used in the lab. All the examples are given using \gls{DVWA} (because of its name) but to get the grade every student should choose another vulnerable application from the list, install  and protect it.

%%{\color{red} Siin on pooleli}
\subsection{Web application firewall and database firewall}


According to the lab scenario the \gls{SQLi} attacks should be stopped before they reach the database (Appendix~\ref{Protecting an Insecure Web Application}). Several proprietary database firewall products as: \emph{Oracle Audit Vault and Database Firewall} and \emph{SecureSphere Database Firewall} can provide the needed functionality. They are not applicable in this lab because they are not free and open source software products.

Only open source database firewall with sufficient functionality such as: web based administrative interface, support for different databases and easy to use was GreenSQL database firewall. However, the development of this product is discontinued  and it can not be downloaded from the vendor homepage~\footnote{GreenSQL -- http://www.greensql.com/}, its source files and pre-build packages for Ubuntu Server 12.04 LTS 64bit are downloadable from the \gls{EITC} lab page~\footnote{\gls{EITC} fork of GreenSQL database -- \url{http://elab.itcollege.ee:8000/Day3/}}. Additionally, the author of this thesis fixed the source code of the firewall to make it compile with newer GNU/C compiler.

For learning the basics of database fire-walling, the enterprise product is not needed and for this lab the open source version of the GreenSQL database firewall was chosen.

Although, the closed source version is still downloadable from the vendor homepage, the future goal is to find an open source replacement for GreenSQL, that is actively developed and has comparable functionality.

The database firewall provides protection only for the database and does not protect against  reflected \gls{XSS} attacks as they are not seen in the database layer. Therefore, this protection is not sufficient and additional filtering in web layer is needed and Web Application Firewall \gls{WAF} to be integrated into the lab scenario.

The commonly used open source \gls{WAF} is ModSecurity~\citep[p.196]{book:practica_intrusion_analysis} from Trustwave SpiderLabs~\footnote{Trustwave SpiderLabs -- \url{https://www.trustwave.com/spiderlabs/}}. However, the ModSecurity itself is a parsing/blocking engine and needs a proper rule set like \gls{OWASP} Core Rule Set Project~\footnote{OWASP ModSecurity Core Rule Set Project -- } to block/log attacks. Although, there are alternatives and the ModSecurity rules are hard to modify (for students), the \gls{EITC} partners and several companies in Estonia are using it according to conversations and discussion between system administrators  from  the pilot lab.

The product supports apache, Enginx and Inernet Information Server web servers~\footnote{Mod Security overview -- \url{http://www.modsecurity.org/projects/modsecurity/index.html}} and is compatible with the web servers chosen for this lab.

Therefore, the Mod Security and \gls{OWASP} Core Rule Set was chosen for this lab.



\section{Development of the e-learning course}
After choosing technological tools the lab materials need to be created~\citep{OppeArenduskeskus2010}. Therefore, the course materials are created, reviewed and run-through by small test group.

\subsection{Authoring the learning material}
Developing learning material is based on learning objectives. Each learning objective must be covered with learning material and everything else should be left out as extra load. Therefore in the development phase the different study materials, assessment tests, audio/video media are authored and integrated into a consistent body to ensure that all the objectives are met~\citep{OppeArenduskeskus2010}.

Authoring the study materials is one of the most time consuming tasks and the amount of developed study materials is too big to include in the appendices. Therefore one sample block - Securing web applications is included in Appendix~\ref{Protecting Web Application Against (D)DOS Attacks} and in Appendix~ \ref{Protecting an Insecure Web Application}. 

The developed learning materials are publicly available  under the \gls{CC-BY-SA} license to guarantee maximum impact in the cyber security field. Therefore the e-learning course is also usable by motivated self-studiers and training companies. 

Learning materials, self-tests, course information and other materials are publicly downloadable~\footnote{\href{http://elab.itcollege.ee:8000/cyber-course/}{Course Syllabus (Õpijuhis)}}.


New materials are stored using the open source revision control system \gls{git} and all changes and commits are publicly available. Additionally, the source code of this thesis is available from a public \gls{git} repository consisting of \LaTeX  source files~\footnote{\href{https://github.com/magavdraakon/margus-thesis.git}{Materials and source code of this thesis}}.


Developed learning materials should follow a consistent style:
%and present also one example of good documentation practice. For system administrators several howto styles exists. However practical hands-on laboratory instructions are designed that pass through using copy paste is possible but gives one working sample. However, this is not enough to pass lab scenario and student must customize own configuration.

%Guiding stile of the lab instruction using style convention: 
First, all variable parts of the text are clearly distinguishable between other text and commands.



Second, all command examples given to the student are highlighted as shown in the following example.

\begin{minted}[frame=lines,framesep=2mm]{bash}
#For changing Out of memory - OOM adjustment score for mysql server
echo "-1000" > /proc/$(pidof mysqld)/oom_score_adj
\end{minted}
%$


The command output is displayed as following:
\small{
\begin{Verbatim}[samepage=true,frame=single,
label=Command output,framesep=2mm,rulecolor=\color{red},commandchars=\\\{\}]
sudent@opiise:~# ps -ef|grep mysqld
root     11290 10905  0 10:27 pts/6    00:00:00 grep --color=auto mysqld
mysql    \fbox{\color{red}29830}    1  0 Apr25 ?        00:05:47 /usr/sbin/mysqld
\end{Verbatim}
%
}

The previous output was produced using this command.
\begin{minted}[frame=lines,framesep=2mm]{bash}
ps -ef|grep mysqld
\end{minted}
\label{code_sample}
%

All study materials should be stored using open formats, like pdf, OpenDocument, MediaWiki markup language, html or utf8 text. The text based materials should be stored into a revision control system like \gls{git} to enable contributing by other lecturers and students as well.

%
%
%\begin{table}[H]
%\centering
%\caption{The learning materials}
%
%\begin{tabular}{|p{5cm}|p{3cm}|p{6cm}|}
%\hline 
%\color{blue}
%Name & \color{blue} Comments  & \color{blue} Location \\ 
%
%\hline
%  \multicolumn{3}{|c|}{Pre-requirement course} \\
%\hline
%Operating system basics & & \\
%\hline
%Basic networking IPv4/IPv6, TCP/IP & & \\
%
%\hline
%GNU/Linux basics  & & \\
%\hline
%Scripting in BASH &  & \\
%\hline
%Scripting in Python & Co authored with Lauri Võsandi & \\
%\hline
%Scripting in PowerShell & Author is Heiki Tähis & \\
%
%
%\hline
%\hline
%  \multicolumn{3}{|c|}{Root services} \\
%
%\hline 
%
%
%Lecture - Configuring NTP service & (Estonian 2012), Learning outcome no XXX & \url{http://goo.gl/toRpw} \\ 
%\hline 
%Practical class - Configuring NTP in Ubuntu  & (Estonian 2012), Learning outcome no XXX , Students improved & \url{https://wiki.itcollege.ee/index.php/NTP_Ubuntus} \\
%\hline 
%Lecture DNS & & \href{http://enos.itcollege.ee/~mernits/infrastruktuur/Interneti%20domeeninimede%20s%c3%bcsteem%20-%20IT%20infra%20loeng.odp}
%{DNS Lecture [OpenDocument]} \\
%\hline
%Practical class - DNS & Co authored with Katrin Loodus  & \href{https://docs.google.com/document/d/1ZeQpPXdVq1C7RQpxQYR0gBB0OBMYB_0g6aFFxs_-fIA/edit}{Configuring DNS [GoogleDocs] } \\
%
%\hline
%\hline
%  \multicolumn{3}{|c|}{Web/File Services} \\
%
%\hline 
% & & \\
%\hline
%
%\hline
% & & \\
%\hline
%\hline
%
%\end{tabular} 
%\label{table:learning_materials}
%\end{table}


\subsection{Course Syllabus}
\label{Course Syllabus}
The course participant will get first information about the course from its syllabus, where all relevant information must be listed, such as: list of learning materials, course schedule, list of labs, list of covered topics, list of exercises and homework, deadlines and grading information. Course Syllabus will be available in the \gls{EITC} study information system and in the lab website~\footnote{Course Syllabus -- \url{http://elab.itcollege.ee:8000/cyber-course/}}.

\subsection{Testing the e-course}
Before experimenting with a larger group of students, all of the course modules should be  tested by co-workers -- lecturers and assistants of the \gls{EITC}. After in-house testing each lab is run-through by a small group of system administrators from different organizations including \gls{EISA}. The evaluation summary for the course is given in Table~\ref{table:desing_develop_evaluation}.
\begin{table}[h]
\centering
\caption{The evaluation of the design and development stage }
{ \small 
\begin{tabular}{|p{6cm}|p{2cm}|p{5cm}|}
\hline 
\color{blue} Evaluation question & \color{blue} Result [1..4] & \color{blue} Comments and references \\
\hline
Does the course have a proper structure? 
& 4  &  The course syllabus has links to the  study program, topics and labs  \\ 
\hline
Does the chosen presentation method of the learning material support the learning outcomes of the course? 
& 4  &  Labs, videos, text and slides are used to support learning outcomes \\ 
\hline

Do the learning materials follow the best practices? 
& 3  &  Some materials need improvement \\ 
\hline
Is all web material available? 
& 4  &  Materials are available in the course syllabus page \\ 
\hline
Does a sufficient course syllabus exist? 
& 4  &  Course Syllabus exists \\ 
\hline 
Does the student need any non-free software for participation?
& 4 &  No, all used software is open source (except PowerShell lab where software is provided by the \gls{EITC}) \\ 
\hline 
Is the course tested before including into the curriculum?
& 4 & The course is tested by 34 students and >80 system administrators \\ 
\hline
Does the course work technically (links, materials)?
& - & Testing is not completed as of now.\\ 
\hline 
\end{tabular} 
}
\label{table:desing_develop_evaluation}
\end{table}


\section{Implementation of the e-learning course}

In the implementation phase, the courses are piloted and feedback from students and other lecturers is gathered. In this phase the learning space and time are arranged, preparing learners for the course~\citep{OppeArenduskeskus2010}.

The author of this thesis piloted each course blocks and collected feedback from the students, made notes with improvement ideas and encouraged students to be as active and motivated as possible.

The first lab was relatively easy because of the topic -- \gls{NTP}. This was intentional because technical details about the virtualization environment and the roles of the lecturer and students in this course are explained to the students during this lab. Therefore the slow start is important for the students to familiarize themselves with the environment and organizational aspects of the course. Thereafter, the studies went smoothly for the students with proper prerequisite skills and knowledge but hard for others. 

The main aspect learned from the implementation phase was that even though this course was hard for most of the students, they were still motivated to learn the content. The preliminary course is mandatory for most of the students and system administrators because  a small group of students with inadequate level may slow down a whole class. 

During the implementation phase the piloting of the course was evaluated using self-assessment and criteria from \gls{ADDIE} model~\citep{OppeArenduskeskus2010}. The results are presented in Table~\ref{table:implementation_evaluation}.

\begin{table}[h]
\centering
\caption{The evaluation of the implementation stage }
{ \small 
\begin{tabular}{|p{6cm}|p{2cm}|p{5cm}|}
\hline 
\color{blue} Evaluation question & \color{blue} Result [1..4] & \color{blue} Comments \\ 
\hline
Are students motivated to be active? 
& 4  &  Active participation of the students was encouraged \\ 
\hline 
Does the lecturer give feedback to the students?
& 4 &  Students got immediate feedback during classes \\ 
\hline 
Did the lecturer collect data during the course on how to improve the course in the future?
& 3 & Data was collected but is not yet used (will be in next course) \\ 
\hline
Is feedback from students collected?
& 4 & Feedback was collected after every course and training. \\ 
\hline 
\end{tabular} 
}
\label{table:implementation_evaluation}
\end{table}

