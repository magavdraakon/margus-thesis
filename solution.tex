\chapter{Solution}
\label{solution}

\lettrine[lraise=0.1, nindent=0em, slope=-.5em]{\color{Violet}A}{lgus} üks lõik enne järgmist pealkirja.

\section{Design and Planning of learning process}

http://www.teaching.utoronto.ca/topics/coursedesign/learning-outcomes/outcomes-objectives.htm




The Design phase contains three steps: First Design Assessments, second choose a course format, third create an instructional strategy.

Design assessments before creating a content - nagu tagant ettepoole minek.

For design an adequate course content two aspects are required, learning objectives and assessment tests. Likewise in software development field the tests should be designed first. Therefore all learning objectives are given with evaluation methods and values.  



\subsubsection{Design of the course content}
All times are given in academic hours or days which equal 8 academic hours.
\begin{enumerate}[label=Hands-on block \arabic*.,leftmargin=*]
  \item Pre-requirements courses
    \begin{enumerate}[label=LAB \arabic*.,leftmargin=*]
  	\item Operating system basics (one day)
  	\item Basic networking IPv4/IPv6, TCP/IP (one day)
  	\item GNU/Linux basics (and OpenBSD/FreeBSD basics) (16h) as describbed in Appendix~\ref{Preliminary course - dpkg based GNU/Linux} on page~\pageref{Preliminary course - dpkg based GNU/Linux}
  	\item Scripting in BASH (2 days)
  	\item Scripting in Python (1.5 days)
  	\item Scripting in PowerShell (1.5 days)
  \end{enumerate}
  \item Root services
  \begin{enumerate}[label=LAB \arabic*.,leftmargin=*]
  	\item NTP (0.5 days)
  	\item DNS (2.5 days)
  	\item DHCP (one day)
  \end{enumerate}
  \item Web/File Services
    \begin{enumerate}[label=LAB \arabic*.,leftmargin=*]
    \item Web server basics - installation of apache2 web server
  	\item Web server security - Protecting Web Application Against
(D)DOS Attacks
  	\item Web server security - securing vulnerable web application using application firewalls (3 days)
  	\end{enumerate}
  	\item Fileserver (Samba3 and Samba4) (0.5 day)
    \item E-mail services (4 days)
    \begin{enumerate}[label=LAB \arabic*.,leftmargin=*]
  		\item SPAM control
	  	\item Virus protection
  		\item MTA's 
	  	\item MDA's
    \end{enumerate}
    \item IP firewalls and IDS/IPS (4 days)
        \begin{enumerate}[label=LAB \arabic*.,leftmargin=*]
  		\item IP firewalls netfilter/iptables and packet filter (pf) (2 days)
	  	\item IDS/IPS (2 days)
  		\item NetFlow (kuna seda loetakse CERT.EE abil, siis jääb välja)
		\end{enumerate}
    \item Autentication and authorization (4 days)
        \begin{enumerate}[label=LAB \arabic*.,leftmargin=*]
  		\item LDAP and Samba4 AD
	  	\item Windows and Linux clients with Samba4 AD 
  		\item Web application authentication with Samba4 AD and LDAP
    		\end{enumerate}
    \item GNU/Linux central management with Puppet
        \begin{enumerate}[label=LAB \arabic*.,leftmargin=*]
	  		\item Installation of Puppet using passenger (2 day)
		  	\item Writing puppet recipes (1 day)
    		\end{enumerate}
    	\item Central logging (3 days)
    	    \begin{enumerate}[label=LAB \arabic*.,leftmargin=*]
	  		\item Collecting logs with rsyslog/syslog-ng (1 day)
		  	\item Monitor and analyse log files (1 day)
    		\end{enumerate}
\end{enumerate}


Assessment design contains (goals, lerners, context, assessment)
kõiki õpiväljundeid tuleb testida.
assessments: clearly written, correct punctuation and grammar.

Choosing a course format is a delivery system like medium used to present the course.
- classroom
- e-course
- blended -combines different methods
- 

Instructional strategy
Collection of 
-lectures
-readings
-discussions
-projects
-worksheets
-assessments
-activities
Activities can be divided into five topics
\begin{enumerate}
\item Pre-instructional activities (motivate the students, mida nad pärast paremini teha oskavad, peale seda presenteerida objectives)
\item Content Presentation (consice content - no too many details, examples, )
\item Learner Participation (practice and feedback, )
\item Assessment (lisaks final assesment, practical assesment, attitude assesment  - ask students how thei feel about course )
\item Follow-through activities (review of entire course strategy .. veidi segane)
\end{enumerate}



Two possible e-learning processes are common the self-paced and facilitated/instructor-led \citep[p.~10]{food2011learning}.

Do we use collaboration in learning? 

Do we use  e-tutoring, e-coaching, e-mentoring?

Do we use chronological order or problem based? (one is good for lecturing other for practical classes) Also possible ways is spiral order logical order etc {\color{red} (viidata) }

Is the content sufficient to gain learning outcomes?

Is amount of work and student workload normal? How many academic hours for practical classes and home reading/lectures?

\subsection{Pedagogical view of the e-course}
Different Pedagogical strategies can be used during learning process. First a problem based approach is {\color{red} Pooleli }
Second koostööl põhinev õpe ja kolmas kogukonnal põhinev õpe.

Do we use group-work, wiki, blog and/or some e-learning environment?

Synchronous or asynchronous learning.

\subsection{Planning grading/assessment techniques}
What grading methods are useful for this particular course?

Do we need grade knowledge, skills and {\color{red} Pooleli }

Several assessment methods are used to give feedback and grades for students in e-courses {\color{red} Viide+listile viide }

\begin{itemize}
	\item self-assessment
	\item computer assessment
	\item tutor assessment
	\item peer assessment
\end{itemize}
\subsection{Choosing technological tools}
For choosing technological tools for a course several following aspects need to be considered:

\begin{itemize}
	\item Availability of e-learning course
	\item usability of e-learning course
	\item Support for peer assessment and community based learning
	\item Support for realistic environment (more then one virtual machine for most of the labs)
	\item Standard complacence
\end{itemize}

For maintaining course materials, collaborations between students and lectures several Learning Management System (\gls{LMS}) are used by Estonian higher educational institutes. Incomplete list  of \gls{LMS}'s are Moodle, Blackboard WebCT (retried), CISO Network Academy, IVA, Sakai and Wikiversity. However the \gls{LMS} are useful for collaboration and storing learning material they can not be used as primary platform for new e-learning course because some grave limitations as: Firstly, the laboratories need virtual machines with administrative privileges in some cases  and supportive infrastructure to simulate Internet around the server. Secondly, only wikiversity and other wiki based systems are suitable for community based learning and peer assessment where students can teach and give feedback to others. Therefore no special \gls{LMS} are used in this course and materials are published in wiki, or in course web page. 






Õpiväljundid on mõeldud õppijate jaoks, et anda neile ülevaade sellest, mida kursuselt oodata ning
millele selle jooksul rõhku panna. Sellest lähtuvalt tuleks nad ka sõnastada vastavalt, st mitte õpetaja
vaid just õppijakeskselt, kirjeldades mitte seda, mis toimub kursuse käigus vaid keskendudes sellele,
mida uut peaks õppija suutma teha kursuse lõpuks.

\begin{enumerate}[label=Learning Objective \arabic*.,leftmargin=*]
\item NTP, DNS ,DHCP
\item Web server, OWASP, DVWA,
\item VPN, SAN, NAS, IDS, IPS
\end{enumerate}

\url{http://blog.spiderlabs.com/2011/01/detecting-malice-with-modsecurity-csrf-attacks.html}



\begin{enumerate}
\item Create a Sample (see on kliendile, testiks ja prototüübiks)
\item Develop the course Materials (peab teadma activities + tagasiside review)
\item Conduct a Run-through (real-time rehearsal testi sõbra peal kogu kursust +  feedback assessment + saab reaalselt teada aja, mis kulub kursuse läbimisele
\end{enumerate}


The implementation phase of the ADDIE Model contains three sub-phases

\begin{enumerate}
\item Train the Instructor
\item Prepare the Learners
\item Arrange the Learning Space
\end{enumerate}

Train the Instructor - course developer is often a trainer too but some cases you need more people to train


Development:
Authoring
Media creation / integration / production
Prototyping
Processing
Quality Assurance

\section{Development of the e-learning course}

\subsection{Authoring the learning material}
Developing learning material is based on learning objectives. Each learning objective must be covered with learning material and everything else should left out as extra load for student. Therefore in the development phase the authoring of different study materials, assessment tests, audio/video media and integration of all artefacts into consistent body with one reason: ensure that all objectives are met.

Authoring the study materials is one of most time consuming tasks and amount of developed study materials is too big even for appendices. Therefore one sample block - Securing web applications are included in Appendix~\ref{Protecting Web Application Against (D)DOS Attacks} and in Appendix~ \ref{Protecting an Insecure Web Application} 

Developed learning material are publicly available and under (\gls{CC-BY-SA}) license to guarantee maximum impact on field. Therefore is acceptable that people with will and motivation may self-study using this e-learning course. Moreover, the private training companies can use those courses to train target group.

Learning materials main download page \url{http://elab.itcollege.ee:8000/cyber-course/} {\color{red} Siia linki vaja kõik materjalid koondada.}

Some never materials are developed using open source revision control system \gls{git} and all changes and commits are publicly available. Moreover, source code of \LaTeX  files and as well other files are public.

Materials and thesis source code \url{https://github.com/magavdraakon/margus-thesis.git}



\url{http://www.grayharriman.com/ADDIE.htm}

Developed learning material should follow consistent style and present also one example of good documentation practice. For system administrators several howto styles exists. However practical hands-on laboratory instructions are designed that pass through using copy paste is possible but gives one working sample. However, this is not enough to pass lab scenario and student must customize own configuration.

Guiding stile of the lab instruction using style convention: First, all variable parts of the text are clearly differs from other text and command. Second, all commands given by student are highlighted, and variable parts embossed as seen in followed command.


\begin{minted}[frame=lines,framesep=2mm]{bash}
#For changing Out of memory - OOM adjustment score for mysql server
echo "-1000" > /proc/$(pidof mysqld)/oom_score_adj
\end{minted}
%$



Sample sample: Finding a proccess ID of the mysql server proccess.

\begin{minted}[frame=lines,framesep=2mm]{bash}
ps -ef|grep mysqld
\end{minted}
\label{code_sample}
%
\small{
\begin{Verbatim}[frame=single,
label=Command output,framesep=2mm,rulecolor=\color{red},commandchars=\\\{\}]
sudent@opiise:~# ps -ef|grep mysqld
root     11290 10905  0 10:27 pts/6    00:00:00 grep --color=auto mysqld
mysql    \fbox{\color{red}29830}    1  0 Apr25 ?        00:05:47 /usr/sbin/mysqld
\end{Verbatim}
%
}

All study should stored in open formats, like pdf, OpenDocument , MediaWiki markup language, html, utf8 text. Original editable source files for generating pdf, images should be publicly accessible. Moreover, text based materials should stored into version control system like \gls{git} to enable contributing for other lecturers and students as well.



\begin{table}[H]
\centering
\caption{The learning materials}

\begin{tabular}{|p{5cm}|p{3cm}|p{6cm}|}
\hline 
\color{blue}
Name & \color{blue} Comments  & \color{blue} Location \\ 

\hline
  \multicolumn{3}{|c|}{Pre-requirement course} \\
\hline
Operating system basics & & \\
\hline
Basic networking IPv4/IPv6, TCP/IP & & \\

\hline
GNU/Linux basics (and OpenBSD/FreeBSD basics)  & & \\
\hline
Scripting in BASH &  & \\
\hline
Scripting in Python & Co authored with Lauri Võsandi & \\
\hline
Scripting in PowerShell & Author is Heiki Tähis & \\


\hline
\hline
  \multicolumn{3}{|c|}{Root services} \\

\hline 


Lecture - Configuring NTP service & (Estonian 2012), Learning outcome no XXX & \url{http://goo.gl/toRpw} \\ 
\hline 
Practical class - Configuring NTP in Ubuntu  & (Estonian 2012), Learning outcome no XXX , Students improved & \url{https://wiki.itcollege.ee/index.php/NTP_Ubuntus} \\
\hline 
Lecture DNS & & \href{http://enos.itcollege.ee/~mernits/infrastruktuur/Interneti%20domeeninimede%20s%c3%bcsteem%20-%20IT%20infra%20loeng.odp}
{DNS Lecture [OpenDocument]} \\
\hline
Practical class - DNS & Co authored with Katrin Loodus  & \href{https://docs.google.com/document/d/1ZeQpPXdVq1C7RQpxQYR0gBB0OBMYB_0g6aFFxs_-fIA/edit}{Configuring DNS [GoogleDocs] } \\

\hline
\hline
  \multicolumn{3}{|c|}{Web/File Services} \\

\hline 
 & & \\
\hline

\hline
 & & \\
\hline
\hline
  \multicolumn{3}{|c|}{E-mail service} \\
\hline 
 & & \\
\hline

\hline
 & & \\

\hline
\hline
  \multicolumn{3}{|c|}{IP firewalls and IDS/IPS} \\
\hline 
 & & \\
\hline

\hline
 & & \\

\hline
\hline
  \multicolumn{3}{|c|}{Autentication and authorization} \\
\hline 
 & & \\
\hline

\hline
 & & \\

\hline
 & & \\
\hline
\end{tabular} 
\label{table:learning_materials}
\end{table}



\subsection{Online tests and laboratory scenarios}

\subsection{Learning objectives}

\subsection{Choosing vulnerable web application}

The main need for vulnerable web application comes from scenario: Each student installs vulnerable system and must stop basic attacks without reprogramming a web application to reflect usual system administrator's work,  install needed applications and secure them.

Although ready made virtual appliances can be used to install vulnerable web application the
system administrator should be able to install vulnerable application itself to understood a main architecture of web application to choose best protection methods. Therefore chosen application should be free and open source, easily installable, implement at leas stored and reflected \gls{XSS} and several injection type attacks like \gls{SQLi} (usual and blind) and \gls{CSRF}.

\subsubsection{WebGoat}
WebGoat is a free, open source insecure J2EE web application designed to teach web application security lessons.  However the WebGoat is one of the best application for teaching the installation and J2EE requirement is not suitable for system administrators with lesser skills even authors did installation script it hides too many steps valuable for students.


\subsubsection{Damn Vulnerable Web Application}
The Damn Vulnerable Web Application \gls{DVWA} is web application with several vulnerabilities to be suitable for testing several security vulnerabilities and tools. However, the tool does not implement all \gls{OWASP} top ten attacks the most relevant are presented. The tool is written using PHP/MySQL which are taught for all \gls{EITC} students and known also by system administrators. Moreover the tool is designed for learning and students can choose difficulty level of exploiting \citep{website:dvwa}.

Although the progam is easy to install. Integrated stdy materials, choosable difficulty level the level of vulnerability coverage is not best but sufficient.

\subsubsection{NOWASP (Mutillidae)}
NOWASP (Mutillidae) Web Pen-Test Practice Application is a free, open source vulnerable web-application for labs, security enthusiast, classrooms, \gls{CTF}, and vulnerability assessment tool targets. \citep{website:Mutillidae}


Pros: Videos, study materials, good support for \gls{OWASP} top ten.
\subsubsection{SQLol}
SQLol is free and open source web application designed to test \gls{SQLi} type injectons and complatible with \gls{MySQL}, gls{PostgreSQL} and uses \gls{PHP}.
cons: only supports \gls{SQLi}
pros: comprehensive  \gls{SQLi} study materials.

 This application was released at at Austin Hackers Association meeting 0x3f by Daniel “unicornFurnace” Crowley of Trustwave Holdings, Inc. – Spider Labs.
Link: https://github.com/SpiderLabs/SQLol



Because of defensive nature on developed course only simple vulnerabilities are needed to demonstrate a problem and all vulnerable applications are suitable for installing. To create diversity all applications what can installed easily can be used in lab. All examples are given using \gls{DVWA} but to get graded every student should choose another



\gls{CSRF}
Protection using mod\_security
\url{http://blog.spiderlabs.com/2011/01/detecting-malice-with-modsecurity-csrf-attacks.html}

\gls{XSS}
{
\color{red} *Kui avastad, kust on malicious script included siis saab esitada avalduse, et see domeen/virtualhost kinni pandaks...(see võiks olla tulevikus osa laborist.

http://www.dcortesi.com/blog/2009/04/11/twitter-stalkdaily-worm-postmortem/
}




\subsection{Technical implementation of the e-course}

\LaTeX Minted muutujad jne


\subsubsection{The Environment of Distance Study}
\label{The Environment of Distance Study}
The Environment of Distance Study...

TODO pilt virtuaalaborite süsteemi kontekstist
\subsubsection{Random tags}
\begin{itemize}
	\item normal traffic generator
	\item malicious traffic generator
	\item availability monitor (for grading)
\end{itemize}

\subsubsection{Virtualization Layer}
Siin libvirdist
\subsubsection{Web Application Layer}
Siin Ruby on Rails raamistikust ja veebirakendusest
\subsubsection{Architecture of Distance Laboratory}
Siin räägin üldisest disainist ja allsüsteemidest
\
\begin{figure}[ht]
\centering
\includegraphics[width=0.8\textwidth]{architecture.png}
\caption{Architecture of Distance Laboratory}
\label{fig:Architecture of Distance Laboratory}
\end{figure}
\

\subsubsection{Security Aspects of Distance Laboratory}

\subsection{Testing the e-course}

\section{Piloting the course}

\subsection{Organizational role}

\subsection{Social role}

\subsection{Pedagogical role}
