\chapter{Introduction}
\label{Introduction}


Siin peatükis räägin EIKs õpetatavatest õppekavadest ja ainetest. Kirjeldan lühidalt süsteemide administreerimise hetkeseisu EIKs, Eestis ja regioonis ja maailmas. Kirjeldan vajadust anda IT süsteemide administreerijatele küberkaitse alaseid teadmiseid.
The Estonian Information Technology College \gls{EITC} is the leading IT institution of applied higher education in Estonia. \cite{EITC} \gls{EITC} is maintained by \gls{HITSA}

Toomase kirjale viide \gls{EISA}
\begin{itemize}
	\item Kirjeldan lühidalt hetkeolukorda
	\item Kirjeldan lühidalt hetkel olevat probleemi (IT adminnidele ei õpetata kuidas teenuseid turvata)
	\item Näitan, et praktilise töö osakaal peaks olemas suurem
	\item Praktilise töö mahtu saab suurendada, kui muuta kodutöö praktiliseks tööks distance lab abil
	
\end{itemize}

Lõputöö eesmärgid
\begin{itemize}
	\item Leida võimalus suurendada praktilise töö mahtu it administreerijate õppekavas
	\item Suurendada küberkaitse alaseid praktiliste oskuste andmist õppekavas
	\item Rakendada kaugtöölaborit praktilise ja mängulise õppe andmisel
\end{itemize}


Lõputöö oodatavad tulemused
\begin{itemize}
	\item Õppekava ainete analüüs arvestades küberkaitse temaatika suurenemist
	\item Loodud e-kursused ja õpiobjektid, mis on rakendatavad nii tasemeõppes, kui ka täiendusõppes
	\item Rakendatud kaugtöölabor loodud õppe toetamiseks
	\item Loodud metoodika õppejõule
	\item kogutud tagasiside õppuritelt ja õppejõududelt
\end{itemize}