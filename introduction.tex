\chapter{Introduction}
\label{Introduction}

 
\begin{quote}
Genuine cybersecurity should not be seen as an additional cost, but as an enabler, guarding our entire digital way of life. --- Toomas Hendrik Ilves
\end{quote}

\lettrine[lraise=0.1, nindent=0em, slope=-.5em]{\color{Violet}C}{ecurity} of the Information and Communication Technology (\gls{ICT}) systems is important part of digital life of society. People connected with global network managed to implement a new digital way for cooperation and trust. E-services and ITC devices are used almost everywhere and countries have \gls{ICT} infrastructure which enables digital way of life.
Cyber security field is rapidly growing and need for highly educated and security aware ITC specialist is increasing. Due to increased malicious activities in the Internet the education in ITC field should emphasize knowledge and skills in cyber security field.

Growth of the Internet connected services and networked infrastructure are collateral to magnitude of possible damage what cyber attack can do to the country. Several countries developed cyber security strategies and -implementation plans to deal with a problem.
Estonia developed strategy plan on 2008 \citep{Strategy2008} and goverment have plans to update strategy at end of 2013 \citep{StrategyProposal2013}. However the strategi states that implementation depends on highly educated IT professionals (TODO viide) and Estonian IT sector has cap between need and possibilities.

Tallinn University of Technology (\gls{TUT}) and University of Tartu (\gls{UT})

The Estonian IT College (\gls{EITC}) is focused on applied higher education. Study programs in \gls{EITC} include development, administration and system analysis. Study program focuses on practical learning by doing approach to ITC subjects. The students gain skills and knowledge as they progress through curricula lectures and practical classes. The studies include lectures, practical classes and independent work as homework. By and large, one subject is divided as follows: 25\% lectures, 25\% practical classes and 50\% homework which is mostly working with materials (books, web articles). In the worst case, practical work constitutes only 25\% and takes place at \gls{EITC} computer classes. Students are not interested in learning mere theory. The formulas are not seen necessary nor linked to their study area or future job. Theory that is not used will be forgotten quickly. In a few years students won't even remember if a specific topic was covered or not. Applied education should introduce practical approach and learning by doing. In ITC field practical classes and practical homework is the key to achieving acceptable results. Students appreciate learning IT system administration by building complex systems using virtual distance laboratory environment and gaining practical experience during this process following learning-by-doing study model.
The problem with practical classes in the cyber field that the practical approach requires hardware, software and preconfigured laboratory environment. However, the  distance laboratory environment can be developed in a manner  that practical classes can be accessed from any Internet connected computer using web browser. Moreover, using virtual and game-like environments is a contemporary approach for teaching IT System administration and programming focusing to the cyber security requirements. The outcome of the developed environment and contemporary teaching methods are students with increased motivation and skills in the cyber field. The tasks given to the students for solving have to be associated with the practical applicability of the theoretical knowledge.

Developed systems allows to  increase of the proportion of practical work by converting independent and theoretical homework to  the practical classes. The quality of studies will improve due to increased amount of practical hands-on classes. The developed system uses only open source components and released using open source compatible MIT licence. The development and hardware are funded by European Social Fund (\gls{ESF}). We expect to have results of using virtual laboratory system to be available in 2013.




The Estonian Information Technology College \gls{EITC} is the IT institution of applied higher education in Estonia and focuses only to the IT field.  \cite{EITC} \gls{EITC} is maintained by Information Technology Foundation for Education (\gls{ITFE}) (In Estonian  \gls{HITSA})

The popularity of the cyber security related subjects in information and communications technology \gls{ICT} curricula is growing, moreover being a "student's magnet" for higher educational institutes.\citep{CyberIsHot}

Now-days a learning by doing approach is accepted way to gain new skills and knowledge in cyber security field. This thesis keskendub development of the practical hands-on e-course for system administrators in \gls{EITC}.


  


Siin peatükis räägin EIKs õpetatavatest õppekavadest ja ainetest. Kirjeldan lühidalt süsteemide administreerimise hetkeseisu EIKs, Eestis ja regioonis ja maailmas. Kirjeldan vajadust anda IT süsteemide administreerijatele küberkaitse alaseid teadmiseid.





Toomase kirjale viide \gls{EISA}


\section{Main Problems}
\begin{itemize}
\item Private companies and governmental sector need for security aware professional systems administrators are increasing.
\item System administrators study needs more practical approach and cyber defese related technical hands-on practical classes.
\item Studying cyber defence should be more attractive and playful for students.
\item IT System administrators in Estonian public sector are often self studied  and do not have higher education in ICT field. Moreover the knowledge needed to protect their IT systems is lesser than industry needs.
\item Private companies offer vendor based courses which often do not provide enough field related knowledge and do not give broader view to the problem.
\end{itemize}

\section{Main Objectives}
Considering to main problems
\begin{itemize}
	\item To develop technical hands-on practical class material in IT system defense field
	\item Suurendada küberkaitse alaseid praktiliste oskuste andmist õppekavas
	\item Rakendada kaugtöölaborit praktilise ja mängulise õppe andmisel
\end{itemize}


\begin{itemize}
	\item Õppekava ainete analüüs arvestades küberkaitse temaatika suurenemist
	\item Loodud e-kursused ja õpiobjektid, mis on rakendatavad nii tasemeõppes, kui ka täiendusõppes
	\item Rakendatud kaugtöölabor loodud õppe toetamiseks
	\item Loodud metoodika õppejõule
	\item kogutud tagasiside õppuritelt ja õppejõududelt
\end{itemize}

Ülesande püstitus.

Veel viiteid uurida 
\url{http://www.emergencymgmt.com/safety/4-Priorities-Improving-Cybersecurity-US.html}\\
\url{http://www.umuc.edu/grad/gradprograms/csec.cfm}
\\
\url{http://www.poly.edu/academics/programs/cybersecurity-ms/curriculum}\\
\url{http://cms.montgomerycollege.edu/EDU/Plain.aspx?id=13043}\\
\url{http://cybersecurity.byu.edu/}\\
\url{http://gcn.com/articles/2010/07/09/cyber-command-panel-afcea-symposium.aspx}\\
\url{http://www.emergencymgmt.com/training/Cybersecurity-Curriculum-University-Maryland-Students.html}\\

\par Peatükkide jaotus