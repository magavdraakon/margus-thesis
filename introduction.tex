\chapter{Introduction}
\label{Introduction}
\lettrine[lraise=0.1, nindent=0em, slope=-.5em]{\color{Violet}N}{owdays} a learning by doing approach is accepted way to gain new skills and knowledge in cyber security field. This thesis keskendub development of the practical hands-on e-course for system administrators in \gls{EITC}.

Growth of the Internet connected services and networked infrastructure are collateral to magnitude of possible damage what cyber attack can do to the country. Several countries developed cyber security strategies and -implementation plans.
Estonia developed strategy plan on 2008 \citep{Strategy2008} and goverment have plans to update strategy at end of 2013 \citep{StrategyProposal2013}
  
  
The popularity of the cyber security related subjects in information and communications technology \gls{ICT} curricula is growing, moreover being a "student's magnet" for higher educational institutes.\citep{CyberIsHot}
  


Siin peatükis räägin EIKs õpetatavatest õppekavadest ja ainetest. Kirjeldan lühidalt süsteemide administreerimise hetkeseisu EIKs, Eestis ja regioonis ja maailmas. Kirjeldan vajadust anda IT süsteemide administreerijatele küberkaitse alaseid teadmiseid.
The Estonian Information Technology College \gls{EITC} is the leading IT institution of applied higher education in Estonia. \cite{EITC} \gls{EITC} is maintained by \gls{HITSA}




Toomase kirjale viide \gls{EISA}


\begin{itemize}
	\item Kirjeldan lühidalt hetkeolukorda
	\item Kirjeldan lühidalt hetkel olevat probleemi (IT adminnidele ei õpetata kuidas teenuseid turvata)
	\item Näitan, et praktilise töö osakaal peaks olemas suurem
	\item Praktilise töö mahtu saab suurendada, kui muuta kodutöö praktiliseks tööks distance lab abil
	
\end{itemize}

Goals
\begin{itemize}
	\item Leida võimalus suurendada praktilise töö mahtu it administreerijate õppekavas
	\item Suurendada küberkaitse alaseid praktiliste oskuste andmist õppekavas
	\item Rakendada kaugtöölaborit praktilise ja mängulise õppe andmisel
\end{itemize}


Expected artefacts
\begin{itemize}
	\item Õppekava ainete analüüs arvestades küberkaitse temaatika suurenemist
	\item Loodud e-kursused ja õpiobjektid, mis on rakendatavad nii tasemeõppes, kui ka täiendusõppes
	\item Rakendatud kaugtöölabor loodud õppe toetamiseks
	\item Loodud metoodika õppejõule
	\item kogutud tagasiside õppuritelt ja õppejõududelt
\end{itemize}

Ülesande püstitus.

Veel viiteid uurida 
\url{http://www.emergencymgmt.com/safety/4-Priorities-Improving-Cybersecurity-US.html}\\
\url{http://www.umuc.edu/grad/gradprograms/csec.cfm}
\\
\url{http://www.poly.edu/academics/programs/cybersecurity-ms/curriculum}\\
\url{http://cms.montgomerycollege.edu/EDU/Plain.aspx?id=13043}\\
\url{http://cybersecurity.byu.edu/}\\
\url{http://gcn.com/articles/2010/07/09/cyber-command-panel-afcea-symposium.aspx}\\
\url{http://www.emergencymgmt.com/training/Cybersecurity-Curriculum-University-Maryland-Students.html}\\

\par Peatükkide jaotus