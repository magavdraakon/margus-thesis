\chapter{Introduction}
\label{Introduction}

 
\begin{quote}
Genuine cybersecurity should not be seen as an additional cost, but as an enabler, guarding our entire digital way of life. --- Toomas Hendrik Ilves
\end{quote}

\lettrine[lraise=0.1, nindent=0em, slope=-.5em]{\color{Violet}T}{he} role of the cyber security is fundamental for entire digital lifestyle. Information society relies on trust to the information and communications technology (\gls{ICT}) systems. However every system needs maintenance and care from highly skilled technicians who have sufficient knowledge for securing complicated networks and services. 

This thesis focuses to practical hands-on e-course for system administrators who are protecting citizen’s everyday digital life. Developed laboratories will be used to teach IT System Administration students in Estonian IT College (\gls{EITC}) and also in continuous education field. Moreover, all study materials are released under Creative Commons Attribution-ShareAlike 3.0 Unported  (\gls{CC-BY-SA}) license and developed supportive software are distributed under \gls{MIT} licence and can be used by any institutions or interested parties.

The cyber security field is rapidly growing and need for highly educated and security aware \gls{ICT} specialist is increasing. Due to increased malicious activities in the Internet the education in gls{ICT} field should emphasize knowledge and skills in cyber security field.

Growth of the Internet connected services and networked infrastructure are collateral to magnitude of possible damage what cyber attack can do to the country. Several countries developed cyber security strategies and -implementation plans to deal with a problem.

Estonia developed strategy plan on 2008 \citep{Strategy2008} and goverment have plans to update strategy at end of 2013 \citep{StrategyProposal2013}. However the strategi states that implementation depends on highly educated IT professionals {\color{red}(TODO viide)} and Estonian IT sector has cap between need and possibilities.

Tallinn University of Technology (\gls{TUT}) and University of Tartu (\gls{UT}) have joint Cyber Security Curricula to deal with the problem. The graduates of joint curricula are often dedicated to management, architecture, cryptography field. However every company needs system administrators who are focused on theirs areas and they are not cyber security specialists. Moreover, they often have degree from applied universities or even no degree at all but have good specialised self-studied education.

The Estonian IT College (\gls{EITC}) is focused on applied higher education. Study programs in \gls{EITC} include development, administration and system analysis. IT System administration curricula is open since 2000 {\color{red}(TODO viide)} and are practical and dedicated to technical areas. During ten years a situation in cyber field are changed to more hostile and frictional. The partner organizations of \gls{EITC} who have possibilities to give input to curricula development process stated need for cyber security related skills and knowledge in study subjects.

Fore example in 2009 the Rector of Estonian IT College received letter from \gls{CERT.EE} with the problem:\par

{
\begin{spacing}{1} 
\scriptsize
---Original Message-----

From: CERT.EE\\
Sent: Tuesday, February 10, 2009 3:55 PM\\
To: Rector of Estonian IT College\\
Cc: Head's of Curricula \\
Subject: Continuous education\\
Hello,

We have a problem:\\
There insufficient amount of IT system administrators in local  governmental, state government and small- and mid-size companies.\\
...\\
With Best Regards,\\
CERT.EE Worker\\
---End of Original Message-----
\end{spacing}
}

For full letter (in Estonian) see Appendix ~\ref{Letter from CERT.EE to Rector of Esonian IT College} on page ~\pageref{Letter from CERT.EE to Rector of Esonian IT College}

Specialists and lecturers from \gls{CERT.EE} and from The Estonian IT College decided to develop practical cyber security module for \gls{EITC} which are usable in higher education and also in continuous education.

The existing system administration subjects should also reviewed and changed according security topic. For example the hands-on class for installing web server should contain installation, configuration and defence and mitigation methods for common attacks. 

For system administrators the security should be include to specific topics extending them instead of separate topics. On other hand some subjects should focus to the security, architecture and processes.

The author of this thesis focuses to the practical classes in this project. Authors contribution are developed hands-on labs for the IT infrastructure services e-course.

\section{Main Problems}
The main problem is lack of skilled and security aware system administrators (in Estonia) who can build companies IT infrastructure and do the everyday maintenance of services.

During discussions with partners and companies several sub-problems in current situations where enlisted:
\begin{itemize}
\item Private companies and governmental sector need for security aware professional systems administrators are increasing. Today Estonian \gls{EITC} educational sector do not fill the gap with new graduates because lack of quality and quantity in particular field. 
\item System administrators study needs more practical approach and cyber defence related technical hands-on practical classes to give better practical and technical preparation for graduates.
\item Studying cyber defence should be more attractive and playful for students. Defending the system is not so interesting comparing to offensive activities.
\item IT System administrators in Estonian public sector are often self studied  and some of them do not have higher education in \gls{ICT} field. Moreover the knowledge needed to protect their IT systems is lesser demands of the industry.
\item Private companies offer vendor and product based courses which often do not provide enough field related knowledge and do not give broader view to the problem.
\end{itemize}

\section{Main Objectives}

The main objective of this thesis is to develop practical hands-on e-course "Securing IT Infrastructure Services" focusing to the installation, configuring and securing infrastructure services.

The particular objective can divided to smaller sub-problems and areas.


Considering to main problems analyse requirements for hands-on e-course and choose methodology to implement the course.

Developed systems allows to  increase of the proportion of practical work by converting independent and theoretical homework to  the practical classes. The quality of studies will improve due to increased amount of practical hands-on classes. The developed system uses only open source components and released using open source compatible MIT licence.



\begin{itemize}
	\item Õppekava ainete analüüs arvestades küberkaitse temaatika suurenemist
	\item Loodud e-kursused ja õpiobjektid, mis on rakendatavad nii tasemeõppes, kui ka täiendusõppes
	\item Rakendatud kaugtöölabor loodud õppe toetamiseks
	\item Loodud metoodika õppejõule
	\item kogutud tagasiside õppuritelt ja õppejõududelt
\end{itemize}


{\large My contributions and main ideas to implement in thesis}
\begin{itemize}
\item To implement practical hands-on labs for system administrators
\item To improve quality of graduates using virtual lab environment  to perform realistic laboratory work and increase amount of practical hands-on study and decrease length of the lectures.
\item To improve motivation and increase role of other skilled tutors use \gls{Coding Dojo} methodology known in programmers field to train system administrators in short seminars. I would like to name it \emph{ command dojo} as most of the work are done in command line of the servers.
\item To improve motivation in defence exercisers use the badges as reward/fame markers for students. Badge for securing webserver from SQLi etc. Rewards are shown in user profile and also seen by other students.
\end{itemize}
\par
{\color{red} Kas peaks kirjeldama sissejuhatuses ka peatükkide jaotust? }
 
